% Options for packages loaded elsewhere
% Options for packages loaded elsewhere
\PassOptionsToPackage{unicode}{hyperref}
\PassOptionsToPackage{hyphens}{url}
\PassOptionsToPackage{dvipsnames,svgnames,x11names}{xcolor}
%
\documentclass[
  letterpaper,
  DIV=11,
  numbers=noendperiod,
  oneside]{scrartcl}
\usepackage{xcolor}
\usepackage[left=1in,marginparwidth=2.0666666666667in,textwidth=4.1333333333333in,marginparsep=0.3in]{geometry}
\usepackage{amsmath,amssymb}
\setcounter{secnumdepth}{-\maxdimen} % remove section numbering
\usepackage{iftex}
\ifPDFTeX
  \usepackage[T1]{fontenc}
  \usepackage[utf8]{inputenc}
  \usepackage{textcomp} % provide euro and other symbols
\else % if luatex or xetex
  \usepackage{unicode-math} % this also loads fontspec
  \defaultfontfeatures{Scale=MatchLowercase}
  \defaultfontfeatures[\rmfamily]{Ligatures=TeX,Scale=1}
\fi
\usepackage{lmodern}
\ifPDFTeX\else
  % xetex/luatex font selection
\fi
% Use upquote if available, for straight quotes in verbatim environments
\IfFileExists{upquote.sty}{\usepackage{upquote}}{}
\IfFileExists{microtype.sty}{% use microtype if available
  \usepackage[]{microtype}
  \UseMicrotypeSet[protrusion]{basicmath} % disable protrusion for tt fonts
}{}
\makeatletter
\@ifundefined{KOMAClassName}{% if non-KOMA class
  \IfFileExists{parskip.sty}{%
    \usepackage{parskip}
  }{% else
    \setlength{\parindent}{0pt}
    \setlength{\parskip}{6pt plus 2pt minus 1pt}}
}{% if KOMA class
  \KOMAoptions{parskip=half}}
\makeatother
% Make \paragraph and \subparagraph free-standing
\makeatletter
\ifx\paragraph\undefined\else
  \let\oldparagraph\paragraph
  \renewcommand{\paragraph}{
    \@ifstar
      \xxxParagraphStar
      \xxxParagraphNoStar
  }
  \newcommand{\xxxParagraphStar}[1]{\oldparagraph*{#1}\mbox{}}
  \newcommand{\xxxParagraphNoStar}[1]{\oldparagraph{#1}\mbox{}}
\fi
\ifx\subparagraph\undefined\else
  \let\oldsubparagraph\subparagraph
  \renewcommand{\subparagraph}{
    \@ifstar
      \xxxSubParagraphStar
      \xxxSubParagraphNoStar
  }
  \newcommand{\xxxSubParagraphStar}[1]{\oldsubparagraph*{#1}\mbox{}}
  \newcommand{\xxxSubParagraphNoStar}[1]{\oldsubparagraph{#1}\mbox{}}
\fi
\makeatother

\usepackage{color}
\usepackage{fancyvrb}
\newcommand{\VerbBar}{|}
\newcommand{\VERB}{\Verb[commandchars=\\\{\}]}
\DefineVerbatimEnvironment{Highlighting}{Verbatim}{commandchars=\\\{\}}
% Add ',fontsize=\small' for more characters per line
\usepackage{framed}
\definecolor{shadecolor}{RGB}{241,243,245}
\newenvironment{Shaded}{\begin{snugshade}}{\end{snugshade}}
\newcommand{\AlertTok}[1]{\textcolor[rgb]{0.68,0.00,0.00}{#1}}
\newcommand{\AnnotationTok}[1]{\textcolor[rgb]{0.37,0.37,0.37}{#1}}
\newcommand{\AttributeTok}[1]{\textcolor[rgb]{0.40,0.45,0.13}{#1}}
\newcommand{\BaseNTok}[1]{\textcolor[rgb]{0.68,0.00,0.00}{#1}}
\newcommand{\BuiltInTok}[1]{\textcolor[rgb]{0.00,0.23,0.31}{#1}}
\newcommand{\CharTok}[1]{\textcolor[rgb]{0.13,0.47,0.30}{#1}}
\newcommand{\CommentTok}[1]{\textcolor[rgb]{0.37,0.37,0.37}{#1}}
\newcommand{\CommentVarTok}[1]{\textcolor[rgb]{0.37,0.37,0.37}{\textit{#1}}}
\newcommand{\ConstantTok}[1]{\textcolor[rgb]{0.56,0.35,0.01}{#1}}
\newcommand{\ControlFlowTok}[1]{\textcolor[rgb]{0.00,0.23,0.31}{\textbf{#1}}}
\newcommand{\DataTypeTok}[1]{\textcolor[rgb]{0.68,0.00,0.00}{#1}}
\newcommand{\DecValTok}[1]{\textcolor[rgb]{0.68,0.00,0.00}{#1}}
\newcommand{\DocumentationTok}[1]{\textcolor[rgb]{0.37,0.37,0.37}{\textit{#1}}}
\newcommand{\ErrorTok}[1]{\textcolor[rgb]{0.68,0.00,0.00}{#1}}
\newcommand{\ExtensionTok}[1]{\textcolor[rgb]{0.00,0.23,0.31}{#1}}
\newcommand{\FloatTok}[1]{\textcolor[rgb]{0.68,0.00,0.00}{#1}}
\newcommand{\FunctionTok}[1]{\textcolor[rgb]{0.28,0.35,0.67}{#1}}
\newcommand{\ImportTok}[1]{\textcolor[rgb]{0.00,0.46,0.62}{#1}}
\newcommand{\InformationTok}[1]{\textcolor[rgb]{0.37,0.37,0.37}{#1}}
\newcommand{\KeywordTok}[1]{\textcolor[rgb]{0.00,0.23,0.31}{\textbf{#1}}}
\newcommand{\NormalTok}[1]{\textcolor[rgb]{0.00,0.23,0.31}{#1}}
\newcommand{\OperatorTok}[1]{\textcolor[rgb]{0.37,0.37,0.37}{#1}}
\newcommand{\OtherTok}[1]{\textcolor[rgb]{0.00,0.23,0.31}{#1}}
\newcommand{\PreprocessorTok}[1]{\textcolor[rgb]{0.68,0.00,0.00}{#1}}
\newcommand{\RegionMarkerTok}[1]{\textcolor[rgb]{0.00,0.23,0.31}{#1}}
\newcommand{\SpecialCharTok}[1]{\textcolor[rgb]{0.37,0.37,0.37}{#1}}
\newcommand{\SpecialStringTok}[1]{\textcolor[rgb]{0.13,0.47,0.30}{#1}}
\newcommand{\StringTok}[1]{\textcolor[rgb]{0.13,0.47,0.30}{#1}}
\newcommand{\VariableTok}[1]{\textcolor[rgb]{0.07,0.07,0.07}{#1}}
\newcommand{\VerbatimStringTok}[1]{\textcolor[rgb]{0.13,0.47,0.30}{#1}}
\newcommand{\WarningTok}[1]{\textcolor[rgb]{0.37,0.37,0.37}{\textit{#1}}}

\usepackage{longtable,booktabs,array}
\usepackage{calc} % for calculating minipage widths
% Correct order of tables after \paragraph or \subparagraph
\usepackage{etoolbox}
\makeatletter
\patchcmd\longtable{\par}{\if@noskipsec\mbox{}\fi\par}{}{}
\makeatother
% Allow footnotes in longtable head/foot
\IfFileExists{footnotehyper.sty}{\usepackage{footnotehyper}}{\usepackage{footnote}}
\makesavenoteenv{longtable}
\usepackage{graphicx}
\makeatletter
\newsavebox\pandoc@box
\newcommand*\pandocbounded[1]{% scales image to fit in text height/width
  \sbox\pandoc@box{#1}%
  \Gscale@div\@tempa{\textheight}{\dimexpr\ht\pandoc@box+\dp\pandoc@box\relax}%
  \Gscale@div\@tempb{\linewidth}{\wd\pandoc@box}%
  \ifdim\@tempb\p@<\@tempa\p@\let\@tempa\@tempb\fi% select the smaller of both
  \ifdim\@tempa\p@<\p@\scalebox{\@tempa}{\usebox\pandoc@box}%
  \else\usebox{\pandoc@box}%
  \fi%
}
% Set default figure placement to htbp
\def\fps@figure{htbp}
\makeatother





\setlength{\emergencystretch}{3em} % prevent overfull lines

\providecommand{\tightlist}{%
  \setlength{\itemsep}{0pt}\setlength{\parskip}{0pt}}



 


\KOMAoption{captions}{tableheading}
\makeatletter
\@ifpackageloaded{tcolorbox}{}{\usepackage[skins,breakable]{tcolorbox}}
\@ifpackageloaded{fontawesome5}{}{\usepackage{fontawesome5}}
\definecolor{quarto-callout-color}{HTML}{909090}
\definecolor{quarto-callout-note-color}{HTML}{0758E5}
\definecolor{quarto-callout-important-color}{HTML}{CC1914}
\definecolor{quarto-callout-warning-color}{HTML}{EB9113}
\definecolor{quarto-callout-tip-color}{HTML}{00A047}
\definecolor{quarto-callout-caution-color}{HTML}{FC5300}
\definecolor{quarto-callout-color-frame}{HTML}{acacac}
\definecolor{quarto-callout-note-color-frame}{HTML}{4582ec}
\definecolor{quarto-callout-important-color-frame}{HTML}{d9534f}
\definecolor{quarto-callout-warning-color-frame}{HTML}{f0ad4e}
\definecolor{quarto-callout-tip-color-frame}{HTML}{02b875}
\definecolor{quarto-callout-caution-color-frame}{HTML}{fd7e14}
\makeatother
\makeatletter
\@ifpackageloaded{caption}{}{\usepackage{caption}}
\AtBeginDocument{%
\ifdefined\contentsname
  \renewcommand*\contentsname{Table of contents}
\else
  \newcommand\contentsname{Table of contents}
\fi
\ifdefined\listfigurename
  \renewcommand*\listfigurename{List of Figures}
\else
  \newcommand\listfigurename{List of Figures}
\fi
\ifdefined\listtablename
  \renewcommand*\listtablename{List of Tables}
\else
  \newcommand\listtablename{List of Tables}
\fi
\ifdefined\figurename
  \renewcommand*\figurename{Figure}
\else
  \newcommand\figurename{Figure}
\fi
\ifdefined\tablename
  \renewcommand*\tablename{Table}
\else
  \newcommand\tablename{Table}
\fi
}
\@ifpackageloaded{float}{}{\usepackage{float}}
\floatstyle{ruled}
\@ifundefined{c@chapter}{\newfloat{codelisting}{h}{lop}}{\newfloat{codelisting}{h}{lop}[chapter]}
\floatname{codelisting}{Listing}
\newcommand*\listoflistings{\listof{codelisting}{List of Listings}}
\makeatother
\makeatletter
\makeatother
\makeatletter
\@ifpackageloaded{caption}{}{\usepackage{caption}}
\@ifpackageloaded{subcaption}{}{\usepackage{subcaption}}
\makeatother
\makeatletter
\@ifpackageloaded{sidenotes}{}{\usepackage{sidenotes}}
\@ifpackageloaded{marginnote}{}{\usepackage{marginnote}}
\makeatother
\usepackage{bookmark}
\IfFileExists{xurl.sty}{\usepackage{xurl}}{} % add URL line breaks if available
\urlstyle{same}
\hypersetup{
  pdftitle={Python \& Tooling},
  pdfauthor={malfadly@sdaia.gov.sa},
  colorlinks=true,
  linkcolor={blue},
  filecolor={Maroon},
  citecolor={Blue},
  urlcolor={Blue},
  pdfcreator={LaTeX via pandoc}}


\title{Python \& Tooling}
\usepackage{etoolbox}
\makeatletter
\providecommand{\subtitle}[1]{% add subtitle to \maketitle
  \apptocmd{\@title}{\par {\large #1 \par}}{}{}
}
\makeatother
\subtitle{AI Professionals Bootcamp \textbar{} Week 1}
\author{}
\date{2025-12-15}
\begin{document}
\maketitle


\section{Day 2: Functions + Files + Better
Profiling}\label{day-2-functions-files-better-profiling}

\textbf{Goal:} Turn yesterday's script into \textbf{clean functions} and
produce a \textbf{richer profiling report} for any CSV.

Bootcamp • SDAIA Academy

\begin{center}\rule{0.5\linewidth}{0.5pt}\end{center}

\subsection{Today's Flow}\label{todays-flow}

\begin{itemize}
\tightlist
\item
  \textbf{Session 1 (60m):} Procedural programming with functions
\item
  \emph{Asr Prayer (20m)}
\item
  \textbf{Session 2 (60m):} Strings + formatting → generate Markdown
  reports
\item
  \emph{Maghrib Prayer (20m)}
\item
  \textbf{Session 3 (60m):} Files + \texttt{pathlib} + \texttt{csv} +
  \texttt{json}
\item
  \emph{Isha Prayer (20m)}
\item
  \textbf{Hands-on (120m):} CSV Profiler --- Part 2 (type inference +
  numeric stats)
\end{itemize}

\begin{center}\rule{0.5\linewidth}{0.5pt}\end{center}

\subsection{Learning Objectives}\label{learning-objectives}

By the end of today, you can:

\begin{itemize}
\tightlist
\item
  Write \textbf{reusable functions} (args, defaults, \texttt{*args},
  \texttt{**kwargs}, keyword-only)
\item
  Use built-ins like \texttt{enumerate}, \texttt{zip}, \texttt{sorted},
  \texttt{any}, \texttt{all}
\item
  Generate clean \textbf{Markdown} with f-strings + \texttt{join()}
\item
  Use \texttt{pathlib.Path} for safe paths
\item
  Read CSVs with \texttt{csv.DictReader} and write JSON with
  \texttt{json.dumps()}
\item
  Upgrade your profiler: \textbf{type inference + stats}
\end{itemize}

\begin{center}\rule{0.5\linewidth}{0.5pt}\end{center}

\subsection{GenAI policy reminder (Week
1)}\label{genai-policy-reminder-week-1}

\begin{itemize}
\tightlist
\item
  ✅ Allowed: asking \textbf{clarifying questions} (syntax, errors,
  concepts)
\item
  ❌ Not allowed: generating code, solutions, reports, or summaries
\end{itemize}

If you are stuck: 1. Re-read your own code 2. Use docs (\texttt{help()},
official docs) 3. Ask a human (TA / instructor)

\marginnote{\begin{footnotesize}

This week is about \emph{building your own skills}, not outsourcing
them.

\end{footnotesize}}

\begin{center}\rule{0.5\linewidth}{0.5pt}\end{center}

\subsection{Warm-up (5 minutes)}\label{warm-up-5-minutes}

Open your Day 1 project and run it.

\textbf{Target command (Unix/macOS):}

\begin{Shaded}
\begin{Highlighting}[]
\BuiltInTok{cd}\NormalTok{ \textasciitilde{}/bootcamp/csv{-}profiler}
\ExtensionTok{uv}\NormalTok{ run python main.py}
\end{Highlighting}
\end{Shaded}

\textbf{Target command (Windows PowerShell):}

\begin{Shaded}
\begin{Highlighting}[]
\FunctionTok{cd} \VariableTok{$HOME}\NormalTok{\textbackslash{}bootcamp\textbackslash{}csv{-}profiler}
\NormalTok{uv run python main}\OperatorTok{.}\FunctionTok{py}
\end{Highlighting}
\end{Shaded}

\textbf{Checkpoint:} \texttt{outputs/report.json} and
\texttt{outputs/report.md} are created.

\begin{center}\rule{0.5\linewidth}{0.5pt}\end{center}

\subsection{\texorpdfstring{If imports fail: fix your \texttt{src/}
imports (for
now)}{If imports fail: fix your src/ imports (for now)}}\label{if-imports-fail-fix-your-src-imports-for-now}

If you see
\texttt{ModuleNotFoundError:\ No\ module\ named\ \textquotesingle{}csv\_profiler\textquotesingle{}}:

\textbf{Option A (recommended today): run with \texttt{PYTHONPATH=src}}

\begin{itemize}
\tightlist
\item
  Unix/macOS:
\end{itemize}

\begin{Shaded}
\begin{Highlighting}[]
\VariableTok{PYTHONPATH}\OperatorTok{=}\NormalTok{src }\ExtensionTok{uv}\NormalTok{ run python main.py}
\end{Highlighting}
\end{Shaded}

\begin{itemize}
\tightlist
\item
  Windows PowerShell:
\end{itemize}

\begin{Shaded}
\begin{Highlighting}[]
\VariableTok{$env}\OperatorTok{:}\VariableTok{PYTHONPATH}\OperatorTok{=}\StringTok{"src"}
\NormalTok{uv run python main}\OperatorTok{.}\FunctionTok{py}
\end{Highlighting}
\end{Shaded}

\begin{tcolorbox}[enhanced jigsaw, breakable, opacitybacktitle=0.6, toptitle=1mm, bottomrule=.15mm, arc=.35mm, toprule=.15mm, bottomtitle=1mm, coltitle=black, titlerule=0mm, title=\textcolor{quarto-callout-tip-color}{\faLightbulb}\hspace{0.5em}{Tip}, rightrule=.15mm, leftrule=.75mm, colback=white, left=2mm, opacityback=0, colframe=quarto-callout-tip-color-frame, colbacktitle=quarto-callout-tip-color!10!white]

We'll formalize packaging later. Today we use this simple approach to
keep moving.

\end{tcolorbox}

\begin{center}\rule{0.5\linewidth}{0.5pt}\end{center}

\subsection{Week project progress}\label{week-project-progress}

You already have (Day 1):

\begin{itemize}
\tightlist
\item
  Read a CSV into \texttt{rows:\ list{[}dict{[}str,\ str{]}{]}}
\item
  Compute a \textbf{basic} report (rows/columns + missing count)
\item
  Write \texttt{report.json} and \texttt{report.md}
\end{itemize}

Today you will add:

\begin{itemize}
\tightlist
\item
  Column \textbf{type inference}: \texttt{number} vs \texttt{text}
\item
  Numeric stats: \texttt{min}, \texttt{max}, \texttt{mean},
  \texttt{unique}
\item
  Cleaner Markdown structure (tables + sections)
\end{itemize}

\section{Session 1}\label{session-1}

\subsection{Functions: your unit of
thinking}\label{functions-your-unit-of-thinking}

3:00pm--4:00pm

\begin{center}\rule{0.5\linewidth}{0.5pt}\end{center}

\subsection{Session 1 objectives}\label{session-1-objectives}

\begin{itemize}
\tightlist
\item
  Explain \textbf{why} we refactor into functions
\item
  Write functions with clear inputs/outputs
\item
  Use different parameter styles safely
\item
  Use built-ins that make your code shorter and clearer
\end{itemize}

\begin{center}\rule{0.5\linewidth}{0.5pt}\end{center}

\subsection{Why functions? (in one
sentence)}\label{why-functions-in-one-sentence}

A function lets you \textbf{name a piece of logic} so you can:

\begin{itemize}
\tightlist
\item
  reuse it
\item
  test it
\item
  read it later without re-thinking it
\item
  change it in one place
\end{itemize}

\marginnote{\begin{footnotesize}

In this project, ``profiling'' becomes a set of small functions you can
compose.

\end{footnotesize}}

\begin{center}\rule{0.5\linewidth}{0.5pt}\end{center}

\subsection{Anatomy of a function}\label{anatomy-of-a-function}

\begin{Shaded}
\begin{Highlighting}[]
\KeywordTok{def}\NormalTok{ greet(name: }\BuiltInTok{str}\NormalTok{) }\OperatorTok{{-}\textgreater{}} \BuiltInTok{str}\NormalTok{:}
    \CommentTok{"""Return a friendly greeting."""}
    \ControlFlowTok{return} \SpecialStringTok{f"Hello }\SpecialCharTok{\{}\NormalTok{name}\SpecialCharTok{\}}\SpecialStringTok{!"}
\end{Highlighting}
\end{Shaded}

\begin{itemize}
\tightlist
\item
  \texttt{def} creates the function
\item
  parameters go inside \texttt{(...)}
\item
  the return type after \texttt{-\textgreater{}} is a \textbf{hint} (not
  enforced automatically)
\item
  docstring describes behavior
\end{itemize}

\begin{center}\rule{0.5\linewidth}{0.5pt}\end{center}

\subsection{Side effects vs return
values}\label{side-effects-vs-return-values}

\textbf{Pure-ish} (good for profiling):

\begin{Shaded}
\begin{Highlighting}[]
\KeywordTok{def}\NormalTok{ mean(values: }\BuiltInTok{list}\NormalTok{[}\BuiltInTok{float}\NormalTok{]) }\OperatorTok{{-}\textgreater{}} \BuiltInTok{float}\NormalTok{:}
    \ControlFlowTok{return} \BuiltInTok{sum}\NormalTok{(values) }\OperatorTok{/} \BuiltInTok{len}\NormalTok{(values)}
\end{Highlighting}
\end{Shaded}

\textbf{Side effect} (still useful, but keep controlled):

\begin{Shaded}
\begin{Highlighting}[]
\KeywordTok{def}\NormalTok{ write\_text(path: }\BuiltInTok{str}\NormalTok{, text: }\BuiltInTok{str}\NormalTok{) }\OperatorTok{{-}\textgreater{}} \VariableTok{None}\NormalTok{:}
    \BuiltInTok{open}\NormalTok{(path, }\StringTok{"w"}\NormalTok{).write(text)}
\end{Highlighting}
\end{Shaded}

\begin{tcolorbox}[enhanced jigsaw, breakable, opacitybacktitle=0.6, toptitle=1mm, bottomrule=.15mm, arc=.35mm, toprule=.15mm, bottomtitle=1mm, coltitle=black, titlerule=0mm, title=\textcolor{quarto-callout-warning-color}{\faExclamationTriangle}\hspace{0.5em}{Warning}, rightrule=.15mm, leftrule=.75mm, colback=white, left=2mm, opacityback=0, colframe=quarto-callout-warning-color-frame, colbacktitle=quarto-callout-warning-color!10!white]

Try to keep \emph{profiling logic} mostly ``return values''. Keep
\emph{I/O} (reading/writing files) in a small number of places.

\end{tcolorbox}

\begin{center}\rule{0.5\linewidth}{0.5pt}\end{center}

\subsection{Parameter types (most
common)}\label{parameter-types-most-common}

\begin{Shaded}
\begin{Highlighting}[]
\KeywordTok{def}\NormalTok{ add(a: }\BuiltInTok{float}\NormalTok{, b: }\BuiltInTok{float} \OperatorTok{=} \FloatTok{1.0}\NormalTok{) }\OperatorTok{{-}\textgreater{}} \BuiltInTok{float}\NormalTok{:}
    \ControlFlowTok{return}\NormalTok{ a }\OperatorTok{+}\NormalTok{ b}
\end{Highlighting}
\end{Shaded}

Calls:

\begin{Shaded}
\begin{Highlighting}[]
\NormalTok{add(}\DecValTok{1}\NormalTok{, }\DecValTok{2}\NormalTok{)}
\NormalTok{add(}\DecValTok{7}\NormalTok{)}
\NormalTok{add(a}\OperatorTok{=}\DecValTok{1}\NormalTok{, b}\OperatorTok{=}\DecValTok{2}\NormalTok{)}
\NormalTok{add(b}\OperatorTok{=}\DecValTok{2}\NormalTok{, a}\OperatorTok{=}\DecValTok{1}\NormalTok{)}
\end{Highlighting}
\end{Shaded}

\begin{center}\rule{0.5\linewidth}{0.5pt}\end{center}

\subsection{Quick Check}\label{quick-check}

What is printed?

\begin{Shaded}
\begin{Highlighting}[]
\KeywordTok{def}\NormalTok{ add(a, b}\OperatorTok{=}\DecValTok{1}\NormalTok{):}
    \ControlFlowTok{return}\NormalTok{ a }\OperatorTok{+}\NormalTok{ b}

\BuiltInTok{print}\NormalTok{(add(}\DecValTok{10}\NormalTok{))}
\BuiltInTok{print}\NormalTok{(add(}\DecValTok{10}\NormalTok{, }\DecValTok{5}\NormalTok{))}
\end{Highlighting}
\end{Shaded}

. . .

\textbf{Answer:} \texttt{11} then \texttt{15}

\begin{center}\rule{0.5\linewidth}{0.5pt}\end{center}

\subsection{\texorpdfstring{\texttt{*args}: ``many positional
arguments''}{*args: ``many positional arguments''}}\label{args-many-positional-arguments}

\begin{Shaded}
\begin{Highlighting}[]
\KeywordTok{def}\NormalTok{ accumulate(}\OperatorTok{*}\NormalTok{numbers: }\BuiltInTok{float}\NormalTok{) }\OperatorTok{{-}\textgreater{}} \BuiltInTok{float}\NormalTok{:}
\NormalTok{    total }\OperatorTok{=} \FloatTok{0.0}
    \ControlFlowTok{for}\NormalTok{ n }\KeywordTok{in}\NormalTok{ numbers:}
\NormalTok{        total }\OperatorTok{+=}\NormalTok{ n}
    \ControlFlowTok{return}\NormalTok{ total}

\NormalTok{accumulate(}\DecValTok{1}\NormalTok{, }\DecValTok{2}\NormalTok{, }\DecValTok{3}\NormalTok{)     }\CommentTok{\# 6.0}
\NormalTok{accumulate(}\DecValTok{5}\NormalTok{)           }\CommentTok{\# 5.0}
\NormalTok{accumulate()            }\CommentTok{\# 0.0}
\end{Highlighting}
\end{Shaded}

\begin{center}\rule{0.5\linewidth}{0.5pt}\end{center}

\subsection{\texorpdfstring{\texttt{**kwargs}: ``many keyword
arguments''}{**kwargs: ``many keyword arguments''}}\label{kwargs-many-keyword-arguments}

\begin{Shaded}
\begin{Highlighting}[]
\KeywordTok{def}\NormalTok{ double(}\OperatorTok{**}\NormalTok{values: }\BuiltInTok{float}\NormalTok{) }\OperatorTok{{-}\textgreater{}} \BuiltInTok{dict}\NormalTok{[}\BuiltInTok{str}\NormalTok{, }\BuiltInTok{float}\NormalTok{]:}
    \ControlFlowTok{return}\NormalTok{ \{k: v }\OperatorTok{*} \DecValTok{2} \ControlFlowTok{for}\NormalTok{ k, v }\KeywordTok{in}\NormalTok{ values.items()\}}

\NormalTok{double(a}\OperatorTok{=}\DecValTok{1}\NormalTok{, b}\OperatorTok{=}\DecValTok{2}\NormalTok{)  }\CommentTok{\# \{"a": 2, "b": 4\}}
\end{Highlighting}
\end{Shaded}

\marginnote{\begin{footnotesize}

In this week, we'll mostly use \texttt{**kwargs} for optional config
later.

\end{footnotesize}}

\begin{center}\rule{0.5\linewidth}{0.5pt}\end{center}

\subsection{Positional-only and keyword-only
parameters}\label{positional-only-and-keyword-only-parameters}

\begin{Shaded}
\begin{Highlighting}[]
\CommentTok{\# a: positional{-}only}
\CommentTok{\# b: positional{-}or{-}keyword}
\CommentTok{\# c: keyword{-}only}
\KeywordTok{def}\NormalTok{ f(a, }\OperatorTok{/}\NormalTok{, b, }\OperatorTok{*}\NormalTok{, c):}
    \BuiltInTok{print}\NormalTok{(a, b, c)}

\NormalTok{f(}\DecValTok{1}\NormalTok{, }\DecValTok{2}\NormalTok{, c}\OperatorTok{=}\DecValTok{3}\NormalTok{)}
\NormalTok{f(}\DecValTok{1}\NormalTok{, b}\OperatorTok{=}\DecValTok{2}\NormalTok{, c}\OperatorTok{=}\DecValTok{3}\NormalTok{)}
\end{Highlighting}
\end{Shaded}

Why care?

\begin{itemize}
\tightlist
\item
  keyword-only parameters make call sites \textbf{more readable}
\item
  positional-only can protect APIs from accidental misuse
\end{itemize}

\begin{center}\rule{0.5\linewidth}{0.5pt}\end{center}

\subsection{Common built-in functions you'll use
today}\label{common-built-in-functions-youll-use-today}

\begin{itemize}
\tightlist
\item
  Sequence math: \texttt{len}, \texttt{sum}, \texttt{min}, \texttt{max},
  \texttt{all}, \texttt{any}
\item
  Sequence helpers: \texttt{sorted}, \texttt{reversed},
  \texttt{enumerate}, \texttt{zip}
\item
  Iteration: \texttt{range}, \texttt{iter}, \texttt{next}
\end{itemize}

\begin{center}\rule{0.5\linewidth}{0.5pt}\end{center}

\subsection{\texorpdfstring{\texttt{enumerate}: index +
value}{enumerate: index + value}}\label{enumerate-index-value}

Instead of:

\begin{Shaded}
\begin{Highlighting}[]
\NormalTok{i }\OperatorTok{=} \DecValTok{0}
\ControlFlowTok{for}\NormalTok{ line }\KeywordTok{in}\NormalTok{ lines:}
    \BuiltInTok{print}\NormalTok{(i, line)}
\NormalTok{    i }\OperatorTok{+=} \DecValTok{1}
\end{Highlighting}
\end{Shaded}

Use:

\begin{Shaded}
\begin{Highlighting}[]
\ControlFlowTok{for}\NormalTok{ i, line }\KeywordTok{in} \BuiltInTok{enumerate}\NormalTok{(lines):}
    \BuiltInTok{print}\NormalTok{(i, line)}
\end{Highlighting}
\end{Shaded}

\begin{center}\rule{0.5\linewidth}{0.5pt}\end{center}

\subsection{\texorpdfstring{\texttt{zip}: walk multiple lists
together}{zip: walk multiple lists together}}\label{zip-walk-multiple-lists-together}

\begin{Shaded}
\begin{Highlighting}[]
\NormalTok{names }\OperatorTok{=}\NormalTok{ [}\StringTok{"A"}\NormalTok{, }\StringTok{"B"}\NormalTok{, }\StringTok{"C"}\NormalTok{]}
\NormalTok{ages }\OperatorTok{=}\NormalTok{ [}\DecValTok{20}\NormalTok{, }\DecValTok{21}\NormalTok{, }\DecValTok{19}\NormalTok{]}

\ControlFlowTok{for}\NormalTok{ name, age }\KeywordTok{in} \BuiltInTok{zip}\NormalTok{(names, ages):}
    \BuiltInTok{print}\NormalTok{(name, age)}
\end{Highlighting}
\end{Shaded}

\marginnote{\begin{footnotesize}

In the report, \texttt{zip()} can pair column names with stats.

\end{footnotesize}}

\begin{center}\rule{0.5\linewidth}{0.5pt}\end{center}

\subsection{\texorpdfstring{\texttt{sorted}: keep original list
unchanged}{sorted: keep original list unchanged}}\label{sorted-keep-original-list-unchanged}

\begin{Shaded}
\begin{Highlighting}[]
\NormalTok{values }\OperatorTok{=}\NormalTok{ [}\DecValTok{3}\NormalTok{, }\DecValTok{1}\NormalTok{, }\DecValTok{2}\NormalTok{]}
\BuiltInTok{print}\NormalTok{(}\BuiltInTok{sorted}\NormalTok{(values))   }\CommentTok{\# [1, 2, 3]}
\BuiltInTok{print}\NormalTok{(values)           }\CommentTok{\# [3, 1, 2]}
\end{Highlighting}
\end{Shaded}

Also useful with a key:

\begin{Shaded}
\begin{Highlighting}[]
\BuiltInTok{sorted}\NormalTok{(words, key}\OperatorTok{=}\BuiltInTok{len}\NormalTok{)}
\end{Highlighting}
\end{Shaded}

\begin{center}\rule{0.5\linewidth}{0.5pt}\end{center}

\subsection{Map/filter vs
comprehensions}\label{mapfilter-vs-comprehensions}

\textbf{These work:}

\begin{Shaded}
\begin{Highlighting}[]
\BuiltInTok{list}\NormalTok{(}\BuiltInTok{map}\NormalTok{(}\KeywordTok{lambda}\NormalTok{ x: x }\OperatorTok{*} \DecValTok{2}\NormalTok{, [}\DecValTok{1}\NormalTok{, }\DecValTok{2}\NormalTok{, }\DecValTok{3}\NormalTok{]))}
\BuiltInTok{list}\NormalTok{(}\BuiltInTok{filter}\NormalTok{(}\KeywordTok{lambda}\NormalTok{ x: x }\OperatorTok{\%} \DecValTok{2} \OperatorTok{==} \DecValTok{0}\NormalTok{, [}\DecValTok{1}\NormalTok{, }\DecValTok{2}\NormalTok{, }\DecValTok{3}\NormalTok{, }\DecValTok{4}\NormalTok{]))}
\end{Highlighting}
\end{Shaded}

But for readability, prefer:

\begin{Shaded}
\begin{Highlighting}[]
\NormalTok{[x }\OperatorTok{*} \DecValTok{2} \ControlFlowTok{for}\NormalTok{ x }\KeywordTok{in}\NormalTok{ [}\DecValTok{1}\NormalTok{, }\DecValTok{2}\NormalTok{, }\DecValTok{3}\NormalTok{]]}
\NormalTok{[x }\ControlFlowTok{for}\NormalTok{ x }\KeywordTok{in}\NormalTok{ [}\DecValTok{1}\NormalTok{, }\DecValTok{2}\NormalTok{, }\DecValTok{3}\NormalTok{, }\DecValTok{4}\NormalTok{] }\ControlFlowTok{if}\NormalTok{ x }\OperatorTok{\%} \DecValTok{2} \OperatorTok{==} \DecValTok{0}\NormalTok{]}
\end{Highlighting}
\end{Shaded}

\begin{center}\rule{0.5\linewidth}{0.5pt}\end{center}

\subsection{Mini-exercise 1 (5
minutes)}\label{mini-exercise-1-5-minutes}

Write a function:

\begin{Shaded}
\begin{Highlighting}[]
\KeywordTok{def}\NormalTok{ is\_missing(value: }\BuiltInTok{str} \OperatorTok{|} \VariableTok{None}\NormalTok{) }\OperatorTok{{-}\textgreater{}} \BuiltInTok{bool}\NormalTok{:}
    \CommentTok{"""True for empty / null{-}ish CSV values."""}
\NormalTok{    ...}
\end{Highlighting}
\end{Shaded}

Treat these as missing (case-insensitive):

\begin{itemize}
\tightlist
\item
  \texttt{""}
\item
  \texttt{"na"}, \texttt{"n/a"}
\item
  \texttt{"null"}, \texttt{"none"}, \texttt{"nan"}
\item
  whitespace-only strings
\end{itemize}

\textbf{Checkpoint:} \texttt{is\_missing("\ NA\ ")} is \texttt{True}.

\begin{center}\rule{0.5\linewidth}{0.5pt}\end{center}

\subsection{\texorpdfstring{Solution ---
\texttt{is\_missing}}{Solution --- is\_missing}}\label{solution-is_missing}

\begin{Shaded}
\begin{Highlighting}[]
\NormalTok{MISSING }\OperatorTok{=}\NormalTok{ \{}\StringTok{""}\NormalTok{, }\StringTok{"na"}\NormalTok{, }\StringTok{"n/a"}\NormalTok{, }\StringTok{"null"}\NormalTok{, }\StringTok{"none"}\NormalTok{, }\StringTok{"nan"}\NormalTok{\}}

\KeywordTok{def}\NormalTok{ is\_missing(value: }\BuiltInTok{str} \OperatorTok{|} \VariableTok{None}\NormalTok{) }\OperatorTok{{-}\textgreater{}} \BuiltInTok{bool}\NormalTok{:}
    \ControlFlowTok{if}\NormalTok{ value }\KeywordTok{is} \VariableTok{None}\NormalTok{:}
        \ControlFlowTok{return} \VariableTok{True}
\NormalTok{    cleaned }\OperatorTok{=}\NormalTok{ value.strip().casefold()}
    \ControlFlowTok{return}\NormalTok{ cleaned }\KeywordTok{in}\NormalTok{ MISSING}
\end{Highlighting}
\end{Shaded}

\begin{tcolorbox}[enhanced jigsaw, breakable, opacitybacktitle=0.6, toptitle=1mm, bottomrule=.15mm, arc=.35mm, toprule=.15mm, bottomtitle=1mm, coltitle=black, titlerule=0mm, title=\textcolor{quarto-callout-tip-color}{\faLightbulb}\hspace{0.5em}{Tip}, rightrule=.15mm, leftrule=.75mm, colback=white, left=2mm, opacityback=0, colframe=quarto-callout-tip-color-frame, colbacktitle=quarto-callout-tip-color!10!white]

Use \texttt{casefold()} (stronger than \texttt{lower()}) for
case-insensitive checks.

\end{tcolorbox}

\begin{center}\rule{0.5\linewidth}{0.5pt}\end{center}

\subsection{Mini-exercise 2 (6
minutes)}\label{mini-exercise-2-6-minutes}

Write a safe parser:

\begin{Shaded}
\begin{Highlighting}[]
\KeywordTok{def}\NormalTok{ try\_float(value: }\BuiltInTok{str}\NormalTok{) }\OperatorTok{{-}\textgreater{}} \BuiltInTok{float} \OperatorTok{|} \VariableTok{None}\NormalTok{:}
    \CommentTok{"""Return float(value) or None if it fails."""}
\NormalTok{    ...}
\end{Highlighting}
\end{Shaded}

\textbf{Checkpoint:} \texttt{try\_float("3.14")} → \texttt{3.14},
\texttt{try\_float("abc")} → \texttt{None}

\begin{center}\rule{0.5\linewidth}{0.5pt}\end{center}

\subsection{\texorpdfstring{Solution ---
\texttt{try\_float}}{Solution --- try\_float}}\label{solution-try_float}

\begin{Shaded}
\begin{Highlighting}[]
\KeywordTok{def}\NormalTok{ try\_float(value: }\BuiltInTok{str}\NormalTok{) }\OperatorTok{{-}\textgreater{}} \BuiltInTok{float} \OperatorTok{|} \VariableTok{None}\NormalTok{:}
    \ControlFlowTok{try}\NormalTok{:}
        \ControlFlowTok{return} \BuiltInTok{float}\NormalTok{(value)}
    \ControlFlowTok{except} \PreprocessorTok{ValueError}\NormalTok{:}
        \ControlFlowTok{return} \VariableTok{None}
\end{Highlighting}
\end{Shaded}

\begin{center}\rule{0.5\linewidth}{0.5pt}\end{center}

\subsection{Type inference (simple
rule)}\label{type-inference-simple-rule}

For a list of strings:

\begin{itemize}
\tightlist
\item
  ignore missing values
\item
  if \textbf{every remaining} value parses as float → \texttt{number}
\item
  else → \texttt{text}
\end{itemize}

\marginnote{\begin{footnotesize}

This is a safe baseline. Later we can add a ``mostly numeric''
threshold.

\end{footnotesize}}

\begin{center}\rule{0.5\linewidth}{0.5pt}\end{center}

\subsection{Example: infer type}\label{example-infer-type}

\begin{Shaded}
\begin{Highlighting}[]
\KeywordTok{def}\NormalTok{ infer\_type(values: }\BuiltInTok{list}\NormalTok{[}\BuiltInTok{str}\NormalTok{]) }\OperatorTok{{-}\textgreater{}} \BuiltInTok{str}\NormalTok{:}
\NormalTok{    usable }\OperatorTok{=}\NormalTok{ [v }\ControlFlowTok{for}\NormalTok{ v }\KeywordTok{in}\NormalTok{ values }\ControlFlowTok{if} \KeywordTok{not}\NormalTok{ is\_missing(v)]}
    \ControlFlowTok{if} \KeywordTok{not}\NormalTok{ usable:}
        \ControlFlowTok{return} \StringTok{"text"}
    \ControlFlowTok{for}\NormalTok{ v }\KeywordTok{in}\NormalTok{ usable:}
        \ControlFlowTok{if}\NormalTok{ try\_float(v) }\KeywordTok{is} \VariableTok{None}\NormalTok{:}
            \ControlFlowTok{return} \StringTok{"text"}
    \ControlFlowTok{return} \StringTok{"number"}
\end{Highlighting}
\end{Shaded}

\begin{center}\rule{0.5\linewidth}{0.5pt}\end{center}

\subsection{Quick Check}\label{quick-check-1}

If a column has values: \texttt{{[}"1",\ "2",\ "3",\ "x"{]}}

\begin{itemize}
\tightlist
\item
  With the simple rule, is it \texttt{number} or \texttt{text}?
\end{itemize}

. . .

\textbf{Answer:} \texttt{text} (because \texttt{"x"} is not numeric)

\begin{center}\rule{0.5\linewidth}{0.5pt}\end{center}

\subsection{Advanced features (you saw them in the
reference)}\label{advanced-features-you-saw-them-in-the-reference}

You will see these patterns later, but you don't need them for today's
assignment:

\begin{itemize}
\tightlist
\item
  recursion
\item
  closures + \texttt{nonlocal}
\item
  generators (\texttt{yield})
\item
  decorators (\texttt{@something})
\item
  \texttt{global} (usually avoid)
\end{itemize}

\begin{tcolorbox}[enhanced jigsaw, breakable, opacitybacktitle=0.6, toptitle=1mm, bottomrule=.15mm, arc=.35mm, toprule=.15mm, bottomtitle=1mm, coltitle=black, titlerule=0mm, title=\textcolor{quarto-callout-warning-color}{\faExclamationTriangle}\hspace{0.5em}{Warning}, rightrule=.15mm, leftrule=.75mm, colback=white, left=2mm, opacityback=0, colframe=quarto-callout-warning-color-frame, colbacktitle=quarto-callout-warning-color!10!white]

If you don't need a ``fancy tool'', don't use it yet. Write the simplest
working code.

\end{tcolorbox}

\begin{center}\rule{0.5\linewidth}{0.5pt}\end{center}

\subsection{Recursion (rare, but you should recognize
it)}\label{recursion-rare-but-you-should-recognize-it}

A function that calls itself needs:

\begin{itemize}
\tightlist
\item
  a \textbf{base case} (stop condition)
\item
  a \textbf{step} that moves toward the base case
\end{itemize}

\begin{Shaded}
\begin{Highlighting}[]
\KeywordTok{def}\NormalTok{ count\_down(n: }\BuiltInTok{int}\NormalTok{) }\OperatorTok{{-}\textgreater{}} \VariableTok{None}\NormalTok{:}
    \ControlFlowTok{if}\NormalTok{ n }\OperatorTok{\textless{}} \DecValTok{0}\NormalTok{:}
        \ControlFlowTok{return}
    \BuiltInTok{print}\NormalTok{(n)}
\NormalTok{    count\_down(n }\OperatorTok{{-}} \DecValTok{1}\NormalTok{)}
\end{Highlighting}
\end{Shaded}

\marginnote{\begin{footnotesize}

In data work, recursion is uncommon. Loops are usually simpler.

\end{footnotesize}}

\begin{center}\rule{0.5\linewidth}{0.5pt}\end{center}

\subsection{\texorpdfstring{Closures + \texttt{nonlocal} (state without
globals)}{Closures + nonlocal (state without globals)}}\label{closures-nonlocal-state-without-globals}

A nested function can ``remember'' variables from the outer function.

\begin{Shaded}
\begin{Highlighting}[]
\KeywordTok{def}\NormalTok{ make\_counter():}
\NormalTok{    i }\OperatorTok{=} \DecValTok{0}
    \KeywordTok{def}\NormalTok{ inc():}
        \KeywordTok{nonlocal}\NormalTok{ i}
\NormalTok{        i }\OperatorTok{+=} \DecValTok{1}
        \ControlFlowTok{return}\NormalTok{ i}
    \ControlFlowTok{return}\NormalTok{ inc}

\NormalTok{c }\OperatorTok{=}\NormalTok{ make\_counter()}
\NormalTok{c()  }\CommentTok{\# 1}
\NormalTok{c()  }\CommentTok{\# 2}
\end{Highlighting}
\end{Shaded}

\begin{center}\rule{0.5\linewidth}{0.5pt}\end{center}

\subsection{Decorators (wrapping a
function)}\label{decorators-wrapping-a-function}

A decorator takes a function and returns a new function.

\begin{Shaded}
\begin{Highlighting}[]
\KeywordTok{def}\NormalTok{ logged(fn):}
    \KeywordTok{def}\NormalTok{ wrapper(}\OperatorTok{*}\NormalTok{args, }\OperatorTok{**}\NormalTok{kwargs):}
        \BuiltInTok{print}\NormalTok{(}\StringTok{"Calling:"}\NormalTok{, fn.}\VariableTok{\_\_name\_\_}\NormalTok{)}
        \ControlFlowTok{return}\NormalTok{ fn(}\OperatorTok{*}\NormalTok{args, }\OperatorTok{**}\NormalTok{kwargs)}
    \ControlFlowTok{return}\NormalTok{ wrapper}

\AttributeTok{@logged}
\KeywordTok{def}\NormalTok{ add(a, b):}
    \ControlFlowTok{return}\NormalTok{ a }\OperatorTok{+}\NormalTok{ b}
\end{Highlighting}
\end{Shaded}

\begin{center}\rule{0.5\linewidth}{0.5pt}\end{center}

\subsection{\texorpdfstring{\texttt{global} (exists, but avoid
it)}{global (exists, but avoid it)}}\label{global-exists-but-avoid-it}

\begin{Shaded}
\begin{Highlighting}[]
\NormalTok{x }\OperatorTok{=} \DecValTok{0}

\KeywordTok{def}\NormalTok{ f():}
    \KeywordTok{global}\NormalTok{ x}
\NormalTok{    x }\OperatorTok{+=} \DecValTok{1}
    \BuiltInTok{print}\NormalTok{(x)}
\end{Highlighting}
\end{Shaded}

Why avoid?

\begin{itemize}
\tightlist
\item
  makes bugs harder to trace
\item
  breaks testability
\end{itemize}

\begin{center}\rule{0.5\linewidth}{0.5pt}\end{center}

\subsection{Common pitfall: mutable default
arguments}\label{common-pitfall-mutable-default-arguments}

Bad:

\begin{Shaded}
\begin{Highlighting}[]
\KeywordTok{def}\NormalTok{ add\_item(x, items}\OperatorTok{=}\NormalTok{[]):}
\NormalTok{    items.append(x)}
    \ControlFlowTok{return}\NormalTok{ items}
\end{Highlighting}
\end{Shaded}

Better:

\begin{Shaded}
\begin{Highlighting}[]
\KeywordTok{def}\NormalTok{ add\_item(x, items}\OperatorTok{=}\VariableTok{None}\NormalTok{):}
    \ControlFlowTok{if}\NormalTok{ items }\KeywordTok{is} \VariableTok{None}\NormalTok{:}
\NormalTok{        items }\OperatorTok{=}\NormalTok{ []}
\NormalTok{    items.append(x)}
    \ControlFlowTok{return}\NormalTok{ items}
\end{Highlighting}
\end{Shaded}

\begin{center}\rule{0.5\linewidth}{0.5pt}\end{center}

\subsection{(Optional) Generators in one
slide}\label{optional-generators-in-one-slide}

\begin{Shaded}
\begin{Highlighting}[]
\KeywordTok{def}\NormalTok{ counter():}
    \ControlFlowTok{yield} \DecValTok{1}
    \ControlFlowTok{yield} \DecValTok{2}
    \ControlFlowTok{yield} \DecValTok{3}

\ControlFlowTok{for}\NormalTok{ x }\KeywordTok{in}\NormalTok{ counter():}
    \BuiltInTok{print}\NormalTok{(x)}
\end{Highlighting}
\end{Shaded}

Why mention it?

\begin{itemize}
\tightlist
\item
  generators let you process data \textbf{without holding everything in
  memory}
\item
  we will keep today's CSV small → lists are fine
\end{itemize}

\begin{center}\rule{0.5\linewidth}{0.5pt}\end{center}

\subsection{Recap (Session 1)}\label{recap-session-1}

\begin{itemize}
\tightlist
\item
  Functions make your code reusable and readable
\item
  Learn parameter styles (\texttt{*args}, \texttt{**kwargs},
  keyword-only)
\item
  Use built-ins (\texttt{enumerate}, \texttt{zip}, \texttt{sorted}) to
  write less code
\item
  Implement core helpers: \texttt{is\_missing}, \texttt{try\_float},
  \texttt{infer\_type}
\end{itemize}

\section{Asr break}\label{asr-break}

\subsection{20 minutes}\label{minutes}

\textbf{When you return:} open your project folder and keep it ready.

\section{Session 2}\label{session-2}

\subsection{Strings → Markdown reports}\label{strings-markdown-reports}

4:20pm--5:20pm

\begin{center}\rule{0.5\linewidth}{0.5pt}\end{center}

\subsection{Session 2 objectives}\label{session-2-objectives}

\begin{itemize}
\tightlist
\item
  Use f-strings + format specs for readable numbers
\item
  Build Markdown using \textbf{lists of lines + join}
\item
  Produce a report that a human can skim in 30 seconds
\end{itemize}

\begin{center}\rule{0.5\linewidth}{0.5pt}\end{center}

\subsection{Markdown is just text}\label{markdown-is-just-text}

Your strategy:

\begin{enumerate}
\def\labelenumi{\arabic{enumi}.}
\tightlist
\item
  Build \texttt{lines:\ list{[}str{]}}
\item
  \texttt{text\ =\ "\textbackslash{}n".join(lines)\ +\ "\textbackslash{}n"}
\item
  Write to a file
\end{enumerate}

This avoids painful string concatenation.

\begin{center}\rule{0.5\linewidth}{0.5pt}\end{center}

\subsection{f-strings: readable and
powerful}\label{f-strings-readable-and-powerful}

\begin{Shaded}
\begin{Highlighting}[]
\NormalTok{rows }\OperatorTok{=} \DecValTok{1234567}
\NormalTok{missing\_pct }\OperatorTok{=} \FloatTok{0.03456}

\BuiltInTok{print}\NormalTok{(}\SpecialStringTok{f"Rows: }\SpecialCharTok{\{}\NormalTok{rows}\SpecialCharTok{:,\}}\SpecialStringTok{"}\NormalTok{)}
\BuiltInTok{print}\NormalTok{(}\SpecialStringTok{f"Missing: }\SpecialCharTok{\{}\NormalTok{missing\_pct}\SpecialCharTok{:.1\%\}}\SpecialStringTok{"}\NormalTok{)}
\end{Highlighting}
\end{Shaded}

Example output:

\begin{itemize}
\tightlist
\item
  \texttt{Rows:\ 1,234,567}
\item
  \texttt{Missing:\ 3.5\%}
\end{itemize}

\begin{center}\rule{0.5\linewidth}{0.5pt}\end{center}

\subsection{Format specs you'll actually
use}\label{format-specs-youll-actually-use}

\begin{itemize}
\tightlist
\item
  \texttt{:,} → thousands separators
\item
  \texttt{.2f} → 2 decimals
\item
  \texttt{.1\%} → percent with 1 decimal
\item
  \texttt{\textgreater{}10} / \texttt{\textless{}10} → align width
\end{itemize}

Example:

\begin{Shaded}
\begin{Highlighting}[]
\NormalTok{value }\OperatorTok{=} \DecValTok{5} \OperatorTok{/} \DecValTok{3}
\BuiltInTok{print}\NormalTok{(}\SpecialStringTok{f"}\SpecialCharTok{\{}\NormalTok{value}\SpecialCharTok{:\textgreater{}8.2f\}}\SpecialStringTok{"}\NormalTok{)}
\end{Highlighting}
\end{Shaded}

\begin{center}\rule{0.5\linewidth}{0.5pt}\end{center}

\subsection{Building a section
(pattern)}\label{building-a-section-pattern}

\begin{Shaded}
\begin{Highlighting}[]
\NormalTok{lines }\OperatorTok{=}\NormalTok{ []}
\NormalTok{lines.append(}\StringTok{"\# CSV Profiling Report"}\NormalTok{)}
\NormalTok{lines.append(}\StringTok{""}\NormalTok{)}
\NormalTok{lines.append(}\StringTok{"\#\# Summary"}\NormalTok{)}
\NormalTok{lines.append(}\SpecialStringTok{f"{-} Rows: }\SpecialCharTok{\{}\NormalTok{n\_rows}\SpecialCharTok{:,\}}\SpecialStringTok{"}\NormalTok{)}
\NormalTok{text }\OperatorTok{=} \StringTok{"}\CharTok{\textbackslash{}n}\StringTok{"}\NormalTok{.join(lines) }\OperatorTok{+} \StringTok{"}\CharTok{\textbackslash{}n}\StringTok{"}
\end{Highlighting}
\end{Shaded}

\begin{center}\rule{0.5\linewidth}{0.5pt}\end{center}

\subsection{Mini-exercise 3 (6
minutes)}\label{mini-exercise-3-6-minutes}

Create a function that returns a markdown header block:

\begin{Shaded}
\begin{Highlighting}[]
\ImportTok{from}\NormalTok{ datetime }\ImportTok{import}\NormalTok{ datetime}

\KeywordTok{def}\NormalTok{ md\_header(source: }\BuiltInTok{str}\NormalTok{) }\OperatorTok{{-}\textgreater{}} \BuiltInTok{list}\NormalTok{[}\BuiltInTok{str}\NormalTok{]:}
    \CommentTok{"""Return lines for the top of the report."""}
\NormalTok{    ...}
\end{Highlighting}
\end{Shaded}

Must include:

\begin{itemize}
\tightlist
\item
  title line \texttt{\#\ CSV\ Profiling\ Report}
\item
  source file name
\item
  generated time (\texttt{datetime.now().isoformat(timespec="seconds")})
\end{itemize}

\begin{center}\rule{0.5\linewidth}{0.5pt}\end{center}

\subsection{\texorpdfstring{Solution ---
\texttt{md\_header}}{Solution --- md\_header}}\label{solution-md_header}

\begin{Shaded}
\begin{Highlighting}[]
\ImportTok{from}\NormalTok{ datetime }\ImportTok{import}\NormalTok{ datetime}

\KeywordTok{def}\NormalTok{ md\_header(source: }\BuiltInTok{str}\NormalTok{) }\OperatorTok{{-}\textgreater{}} \BuiltInTok{list}\NormalTok{[}\BuiltInTok{str}\NormalTok{]:}
\NormalTok{    ts }\OperatorTok{=}\NormalTok{ datetime.now().isoformat(timespec}\OperatorTok{=}\StringTok{"seconds"}\NormalTok{)}
    \ControlFlowTok{return}\NormalTok{ [}
        \StringTok{"\# CSV Profiling Report"}\NormalTok{,}
        \StringTok{""}\NormalTok{,}
        \SpecialStringTok{f"{-} **Source:** \textasciigrave{}}\SpecialCharTok{\{}\NormalTok{source}\SpecialCharTok{\}}\SpecialStringTok{\textasciigrave{}"}\NormalTok{,}
        \SpecialStringTok{f"{-} **Generated:** \textasciigrave{}}\SpecialCharTok{\{}\NormalTok{ts}\SpecialCharTok{\}}\SpecialStringTok{\textasciigrave{}"}\NormalTok{,}
        \StringTok{""}\NormalTok{,}
\NormalTok{    ]}
\end{Highlighting}
\end{Shaded}

\begin{center}\rule{0.5\linewidth}{0.5pt}\end{center}

\subsection{Markdown tables (quick
pattern)}\label{markdown-tables-quick-pattern}

\begin{Shaded}
\begin{Highlighting}[]
\NormalTok{| Column | Type | Missing | Unique |}
\NormalTok{|{-}{-}{-}|{-}{-}{-}:|{-}{-}{-}:|{-}{-}{-}:|}
\NormalTok{| age | number | 0 (0.0\%) | 12 |}
\end{Highlighting}
\end{Shaded}

Rules:

\begin{itemize}
\tightlist
\item
  the second line is the ``separator''
\item
  align numeric columns with \texttt{-\/-\/-:} (optional but nice)
\end{itemize}

\begin{center}\rule{0.5\linewidth}{0.5pt}\end{center}

\subsection{Mini-exercise 4 (8
minutes)}\label{mini-exercise-4-8-minutes}

Write a function to render the table header:

\begin{Shaded}
\begin{Highlighting}[]
\KeywordTok{def}\NormalTok{ md\_table\_header() }\OperatorTok{{-}\textgreater{}} \BuiltInTok{list}\NormalTok{[}\BuiltInTok{str}\NormalTok{]:}
\NormalTok{    ...}
\end{Highlighting}
\end{Shaded}

It should return:

\begin{itemize}
\tightlist
\item
  header row
\item
  separator row
\end{itemize}

\begin{center}\rule{0.5\linewidth}{0.5pt}\end{center}

\subsection{\texorpdfstring{Solution ---
\texttt{md\_table\_header}}{Solution --- md\_table\_header}}\label{solution-md_table_header}

\begin{Shaded}
\begin{Highlighting}[]
\KeywordTok{def}\NormalTok{ md\_table\_header() }\OperatorTok{{-}\textgreater{}} \BuiltInTok{list}\NormalTok{[}\BuiltInTok{str}\NormalTok{]:}
    \ControlFlowTok{return}\NormalTok{ [}
        \StringTok{"| Column | Type | Missing | Unique |"}\NormalTok{,}
        \StringTok{"|{-}{-}{-}|{-}{-}{-}:|{-}{-}{-}:|{-}{-}{-}:|"}\NormalTok{,}
\NormalTok{    ]}
\end{Highlighting}
\end{Shaded}

\begin{center}\rule{0.5\linewidth}{0.5pt}\end{center}

\subsection{Rendering one row
(pattern)}\label{rendering-one-row-pattern}

\begin{Shaded}
\begin{Highlighting}[]
\KeywordTok{def}\NormalTok{ md\_col\_row(name: }\BuiltInTok{str}\NormalTok{, typ: }\BuiltInTok{str}\NormalTok{, missing: }\BuiltInTok{int}\NormalTok{, missing\_pct: }\BuiltInTok{float}\NormalTok{, unique: }\BuiltInTok{int}\NormalTok{) }\OperatorTok{{-}\textgreater{}} \BuiltInTok{str}\NormalTok{:}
    \ControlFlowTok{return} \SpecialStringTok{f"| \textasciigrave{}}\SpecialCharTok{\{}\NormalTok{name}\SpecialCharTok{\}}\SpecialStringTok{\textasciigrave{} | }\SpecialCharTok{\{}\NormalTok{typ}\SpecialCharTok{\}}\SpecialStringTok{ | }\SpecialCharTok{\{}\NormalTok{missing}\SpecialCharTok{\}}\SpecialStringTok{ (}\SpecialCharTok{\{}\NormalTok{missing\_pct}\SpecialCharTok{:.1\%\}}\SpecialStringTok{) | }\SpecialCharTok{\{}\NormalTok{unique}\SpecialCharTok{\}}\SpecialStringTok{ |"}
\end{Highlighting}
\end{Shaded}

\begin{center}\rule{0.5\linewidth}{0.5pt}\end{center}

\subsection{Mini-exercise 5 (7
minutes)}\label{mini-exercise-5-7-minutes}

Write:

\begin{Shaded}
\begin{Highlighting}[]
\KeywordTok{def}\NormalTok{ md\_bullets(items: }\BuiltInTok{list}\NormalTok{[}\BuiltInTok{str}\NormalTok{]) }\OperatorTok{{-}\textgreater{}} \BuiltInTok{list}\NormalTok{[}\BuiltInTok{str}\NormalTok{]:}
    \CommentTok{"""Turn [\textquotesingle{}a\textquotesingle{},\textquotesingle{}b\textquotesingle{}] into [\textquotesingle{}{-} a\textquotesingle{},\textquotesingle{}{-} b\textquotesingle{}]"""}
\NormalTok{    ...}
\end{Highlighting}
\end{Shaded}

\begin{center}\rule{0.5\linewidth}{0.5pt}\end{center}

\subsection{\texorpdfstring{Solution ---
\texttt{md\_bullets}}{Solution --- md\_bullets}}\label{solution-md_bullets}

\begin{Shaded}
\begin{Highlighting}[]
\KeywordTok{def}\NormalTok{ md\_bullets(items: }\BuiltInTok{list}\NormalTok{[}\BuiltInTok{str}\NormalTok{]) }\OperatorTok{{-}\textgreater{}} \BuiltInTok{list}\NormalTok{[}\BuiltInTok{str}\NormalTok{]:}
    \ControlFlowTok{return}\NormalTok{ [}\SpecialStringTok{f"{-} }\SpecialCharTok{\{}\NormalTok{x}\SpecialCharTok{\}}\SpecialStringTok{"} \ControlFlowTok{for}\NormalTok{ x }\KeywordTok{in}\NormalTok{ items]}
\end{Highlighting}
\end{Shaded}

\begin{center}\rule{0.5\linewidth}{0.5pt}\end{center}

\subsection{Common string ``cleaning''
moves}\label{common-string-cleaning-moves}

When working with CSVs:

\begin{itemize}
\tightlist
\item
  \texttt{strip()} to remove whitespace
\item
  \texttt{casefold()} for robust lowercasing
\item
  \texttt{replace()} to normalize text
\item
  \texttt{split()} to break things apart
\item
  \texttt{join()} to assemble output
\end{itemize}

\begin{center}\rule{0.5\linewidth}{0.5pt}\end{center}

\subsection{Quick Check}\label{quick-check-2}

What is the output?

\begin{Shaded}
\begin{Highlighting}[]
\NormalTok{s }\OperatorTok{=} \StringTok{"  NA  "}
\BuiltInTok{print}\NormalTok{(s.strip().casefold())}
\end{Highlighting}
\end{Shaded}

. . .

\textbf{Answer:} \texttt{na}

\begin{center}\rule{0.5\linewidth}{0.5pt}\end{center}

\subsection{Recap (Session 2)}\label{recap-session-2}

\begin{itemize}
\tightlist
\item
  Build Markdown using \texttt{lines} +
  \texttt{"\textbackslash{}n".join(lines)}
\item
  Use f-string format specs for readable numbers
\item
  Use small rendering helpers (\texttt{md\_header},
  \texttt{md\_table\_header}, \texttt{md\_col\_row})
\end{itemize}

\section{Maghrib break}\label{maghrib-break}

\subsection{20 minutes}\label{minutes-1}

\textbf{When you return:} we'll work with files + \texttt{csv} +
\texttt{json}.

\section{Session 3}\label{session-3}

\subsection{\texorpdfstring{Files + \texttt{pathlib} +
formats}{Files + pathlib + formats}}\label{files-pathlib-formats}

5:40pm--6:40pm

\begin{center}\rule{0.5\linewidth}{0.5pt}\end{center}

\subsection{Session 3 objectives}\label{session-3-objectives}

\begin{itemize}
\tightlist
\item
  Use \texttt{pathlib.Path} instead of fragile string paths
\item
  Read CSV robustly with \texttt{csv.DictReader}
\item
  Write JSON with stable formatting
\item
  Apply safe file-writing habits (\texttt{mkdir}, encoding, newline)
\end{itemize}

\begin{center}\rule{0.5\linewidth}{0.5pt}\end{center}

\subsection{\texorpdfstring{\texttt{pathlib.Path} is your
default}{pathlib.Path is your default}}\label{pathlib.path-is-your-default}

\begin{Shaded}
\begin{Highlighting}[]
\ImportTok{from}\NormalTok{ pathlib }\ImportTok{import}\NormalTok{ Path}

\NormalTok{p }\OperatorTok{=}\NormalTok{ Path(}\StringTok{"data"}\NormalTok{) }\OperatorTok{/} \StringTok{"sample.csv"}
\BuiltInTok{print}\NormalTok{(p.exists())}
\BuiltInTok{print}\NormalTok{(p.suffix)   }\CommentTok{\# ".csv"}
\BuiltInTok{print}\NormalTok{(p.stem)     }\CommentTok{\# "sample"}
\end{Highlighting}
\end{Shaded}

Why?

\begin{itemize}
\tightlist
\item
  cross-platform paths
\item
  easy parent folders
\item
  clearer code
\end{itemize}

\begin{center}\rule{0.5\linewidth}{0.5pt}\end{center}

\subsection{Safe ``write file'' pattern}\label{safe-write-file-pattern}

\begin{Shaded}
\begin{Highlighting}[]
\ImportTok{from}\NormalTok{ pathlib }\ImportTok{import}\NormalTok{ Path}

\KeywordTok{def}\NormalTok{ write\_text(path: }\BuiltInTok{str} \OperatorTok{|}\NormalTok{ Path, text: }\BuiltInTok{str}\NormalTok{) }\OperatorTok{{-}\textgreater{}} \VariableTok{None}\NormalTok{:}
\NormalTok{    path }\OperatorTok{=}\NormalTok{ Path(path)}
\NormalTok{    path.parent.mkdir(parents}\OperatorTok{=}\VariableTok{True}\NormalTok{, exist\_ok}\OperatorTok{=}\VariableTok{True}\NormalTok{)}
\NormalTok{    path.write\_text(text, encoding}\OperatorTok{=}\StringTok{"utf{-}8"}\NormalTok{)}
\end{Highlighting}
\end{Shaded}

\begin{center}\rule{0.5\linewidth}{0.5pt}\end{center}

\subsection{Quick Check}\label{quick-check-3}

Why do we use \texttt{mkdir(parents=True,\ exist\_ok=True)}?

A. It deletes old folders\\
B. It creates folders if missing, and doesn't crash if they exist\\
C. It makes the file smaller

. . .

\textbf{Answer:} B

\begin{center}\rule{0.5\linewidth}{0.5pt}\end{center}

\subsection{\texorpdfstring{CSV reading:
\texttt{DictReader}}{CSV reading: DictReader}}\label{csv-reading-dictreader}

\begin{Shaded}
\begin{Highlighting}[]
\ImportTok{from}\NormalTok{ csv }\ImportTok{import}\NormalTok{ DictReader}
\ImportTok{from}\NormalTok{ pathlib }\ImportTok{import}\NormalTok{ Path}

\KeywordTok{def}\NormalTok{ read\_csv\_rows(path: }\BuiltInTok{str} \OperatorTok{|}\NormalTok{ Path) }\OperatorTok{{-}\textgreater{}} \BuiltInTok{list}\NormalTok{[}\BuiltInTok{dict}\NormalTok{[}\BuiltInTok{str}\NormalTok{, }\BuiltInTok{str}\NormalTok{]]:}
\NormalTok{    path }\OperatorTok{=}\NormalTok{ Path(path)}
    \ControlFlowTok{with}\NormalTok{ path.}\BuiltInTok{open}\NormalTok{(}\StringTok{"r"}\NormalTok{, encoding}\OperatorTok{=}\StringTok{"utf{-}8"}\NormalTok{, newline}\OperatorTok{=}\StringTok{""}\NormalTok{) }\ImportTok{as}\NormalTok{ f:}
        \ControlFlowTok{return}\NormalTok{ [}\BuiltInTok{dict}\NormalTok{(row) }\ControlFlowTok{for}\NormalTok{ row }\KeywordTok{in}\NormalTok{ DictReader(f)]}
\end{Highlighting}
\end{Shaded}

Notes:

\begin{itemize}
\tightlist
\item
  \texttt{newline=""} is recommended for CSV files
\item
  values come as strings → you parse them
\end{itemize}

\begin{center}\rule{0.5\linewidth}{0.5pt}\end{center}

\subsection{Mini-exercise 6 (7
minutes)}\label{mini-exercise-6-7-minutes}

Write a helper to get columns from rows:

\begin{Shaded}
\begin{Highlighting}[]
\KeywordTok{def}\NormalTok{ get\_columns(rows: }\BuiltInTok{list}\NormalTok{[}\BuiltInTok{dict}\NormalTok{[}\BuiltInTok{str}\NormalTok{, }\BuiltInTok{str}\NormalTok{]]) }\OperatorTok{{-}\textgreater{}} \BuiltInTok{list}\NormalTok{[}\BuiltInTok{str}\NormalTok{]:}
\NormalTok{    ...}
\end{Highlighting}
\end{Shaded}

Rules:

\begin{itemize}
\tightlist
\item
  if \texttt{rows} is empty → return \texttt{{[}{]}}
\item
  otherwise use the first row's keys
\end{itemize}

\begin{center}\rule{0.5\linewidth}{0.5pt}\end{center}

\subsection{\texorpdfstring{Solution ---
\texttt{get\_columns}}{Solution --- get\_columns}}\label{solution-get_columns}

\begin{Shaded}
\begin{Highlighting}[]
\KeywordTok{def}\NormalTok{ get\_columns(rows: }\BuiltInTok{list}\NormalTok{[}\BuiltInTok{dict}\NormalTok{[}\BuiltInTok{str}\NormalTok{, }\BuiltInTok{str}\NormalTok{]]) }\OperatorTok{{-}\textgreater{}} \BuiltInTok{list}\NormalTok{[}\BuiltInTok{str}\NormalTok{]:}
    \ControlFlowTok{if} \KeywordTok{not}\NormalTok{ rows:}
        \ControlFlowTok{return}\NormalTok{ []}
    \ControlFlowTok{return} \BuiltInTok{list}\NormalTok{(rows[}\DecValTok{0}\NormalTok{].keys())}
\end{Highlighting}
\end{Shaded}

\begin{center}\rule{0.5\linewidth}{0.5pt}\end{center}

\subsection{JSON writing: stable and
readable}\label{json-writing-stable-and-readable}

\begin{Shaded}
\begin{Highlighting}[]
\ImportTok{import}\NormalTok{ json}
\ImportTok{from}\NormalTok{ pathlib }\ImportTok{import}\NormalTok{ Path}

\KeywordTok{def}\NormalTok{ write\_json(report: }\BuiltInTok{dict}\NormalTok{, path: }\BuiltInTok{str} \OperatorTok{|}\NormalTok{ Path) }\OperatorTok{{-}\textgreater{}} \VariableTok{None}\NormalTok{:}
\NormalTok{    path }\OperatorTok{=}\NormalTok{ Path(path)}
\NormalTok{    path.parent.mkdir(parents}\OperatorTok{=}\VariableTok{True}\NormalTok{, exist\_ok}\OperatorTok{=}\VariableTok{True}\NormalTok{)}
\NormalTok{    text }\OperatorTok{=}\NormalTok{ json.dumps(report, indent}\OperatorTok{=}\DecValTok{2}\NormalTok{, ensure\_ascii}\OperatorTok{=}\VariableTok{False}\NormalTok{) }\OperatorTok{+} \StringTok{"}\CharTok{\textbackslash{}n}\StringTok{"}
\NormalTok{    path.write\_text(text, encoding}\OperatorTok{=}\StringTok{"utf{-}8"}\NormalTok{)}
\end{Highlighting}
\end{Shaded}

\begin{center}\rule{0.5\linewidth}{0.5pt}\end{center}

\subsection{Designing your report schema
(practical)}\label{designing-your-report-schema-practical}

Keep it simple and stable:

\begin{Shaded}
\begin{Highlighting}[]
\NormalTok{report }\OperatorTok{=}\NormalTok{ \{}
  \StringTok{"source"}\NormalTok{: \{}\StringTok{"path"}\NormalTok{: }\StringTok{"..."}\NormalTok{\},}
  \StringTok{"summary"}\NormalTok{: \{}\StringTok{"rows"}\NormalTok{: }\DecValTok{100}\NormalTok{, }\StringTok{"columns"}\NormalTok{: }\DecValTok{8}\NormalTok{\},}
  \StringTok{"columns"}\NormalTok{: \{}
     \StringTok{"age"}\NormalTok{: \{}\StringTok{"type"}\NormalTok{: }\StringTok{"number"}\NormalTok{, }\StringTok{"missing"}\NormalTok{: }\DecValTok{2}\NormalTok{, }\StringTok{"unique"}\NormalTok{: }\DecValTok{31}\NormalTok{, }\StringTok{"min"}\NormalTok{: }\FloatTok{0.0}\NormalTok{, ...\},}
     \StringTok{"city"}\NormalTok{: \{}\StringTok{"type"}\NormalTok{: }\StringTok{"text"}\NormalTok{, }\StringTok{"missing"}\NormalTok{: }\DecValTok{0}\NormalTok{, }\StringTok{"unique"}\NormalTok{: }\DecValTok{5}\NormalTok{, }\StringTok{"top"}\NormalTok{: [\{}\StringTok{"value"}\NormalTok{: }\StringTok{"Riyadh"}\NormalTok{, }\StringTok{"count"}\NormalTok{: }\DecValTok{20}\NormalTok{\}]\},}
\NormalTok{  \}}
\NormalTok{\}}
\end{Highlighting}
\end{Shaded}

\begin{tcolorbox}[enhanced jigsaw, breakable, opacitybacktitle=0.6, toptitle=1mm, bottomrule=.15mm, arc=.35mm, toprule=.15mm, bottomtitle=1mm, coltitle=black, titlerule=0mm, title=\textcolor{quarto-callout-tip-color}{\faLightbulb}\hspace{0.5em}{Tip}, rightrule=.15mm, leftrule=.75mm, colback=white, left=2mm, opacityback=0, colframe=quarto-callout-tip-color-frame, colbacktitle=quarto-callout-tip-color!10!white]

A clear schema makes your CLI and Streamlit app easy later.

\end{tcolorbox}

\begin{center}\rule{0.5\linewidth}{0.5pt}\end{center}

\subsection{Optional: base64 vs pickle (30
seconds)}\label{optional-base64-vs-pickle-30-seconds}

\begin{itemize}
\tightlist
\item
  \texttt{base64}: text encoding of bytes (useful for transport)
\item
  \texttt{pickle}: Python-only serialization
\end{itemize}

\begin{tcolorbox}[enhanced jigsaw, breakable, opacitybacktitle=0.6, toptitle=1mm, bottomrule=.15mm, arc=.35mm, toprule=.15mm, bottomtitle=1mm, coltitle=black, titlerule=0mm, title=\textcolor{quarto-callout-warning-color}{\faExclamationTriangle}\hspace{0.5em}{Warning}, rightrule=.15mm, leftrule=.75mm, colback=white, left=2mm, opacityback=0, colframe=quarto-callout-warning-color-frame, colbacktitle=quarto-callout-warning-color!10!white]

Never unpickle data from someone you don't trust. Pickle can execute
code during loading.

\end{tcolorbox}

\begin{center}\rule{0.5\linewidth}{0.5pt}\end{center}

\subsection{Recap (Session 3)}\label{recap-session-3}

\begin{itemize}
\tightlist
\item
  Use \texttt{Path} for paths + folder creation
\item
  Read CSV with \texttt{DictReader}
\item
  Write JSON with \texttt{json.dumps(...,\ indent=2)}
\item
  Define a clear report schema
\end{itemize}

\section{Isha break}\label{isha-break}

\subsection{20 minutes}\label{minutes-2}

\textbf{When you return:} Hands-on build (Part 2).

\section{Session 4}\label{session-4}

\subsection{Hands-on: CSV Profiler (Part
2)}\label{hands-on-csv-profiler-part-2}

7:00pm--9:00pm

\begin{center}\rule{0.5\linewidth}{0.5pt}\end{center}

\subsection{Hands-on success criteria (what ``done''
means)}\label{hands-on-success-criteria-what-done-means}

By 9:00pm:

\begin{itemize}
\tightlist
\item
  Your profiler detects \texttt{number} vs \texttt{text}
\item
  For numeric columns you compute: \texttt{count}, \texttt{missing},
  \texttt{unique}, \texttt{min}, \texttt{max}, \texttt{mean}
\item
  You generate:

  \begin{itemize}
  \tightlist
  \item
    \texttt{outputs/report.json}
  \item
    \texttt{outputs/report.md} (with a summary + a table + per-column
    details)
  \end{itemize}
\end{itemize}

\begin{center}\rule{0.5\linewidth}{0.5pt}\end{center}

\subsection{Lab rules (to stay
productive)}\label{lab-rules-to-stay-productive}

\begin{itemize}
\tightlist
\item
  Work in pairs (driver / navigator), switch every 15 minutes
\item
  Keep functions small (10--25 lines)
\item
  If stuck \textgreater{} 5 minutes:

  \begin{itemize}
  \tightlist
  \item
    write down the exact error
  \item
    read it out loud
  \item
    then ask the instructor/TA
  \end{itemize}
\end{itemize}

\begin{center}\rule{0.5\linewidth}{0.5pt}\end{center}

\subsection{Task 0 --- Create a ``today branch''
(optional)}\label{task-0-create-a-today-branch-optional}

If you already know Git:

\begin{Shaded}
\begin{Highlighting}[]
\FunctionTok{git}\NormalTok{ checkout }\AttributeTok{{-}b}\NormalTok{ day2}
\end{Highlighting}
\end{Shaded}

If you don't know Git yet: \textbf{skip this}. We'll cover Git later
this week.

\begin{center}\rule{0.5\linewidth}{0.5pt}\end{center}

\subsection{Task 1 --- Add shared helpers (10
minutes)}\label{task-1-add-shared-helpers-10-minutes}

In \texttt{src/csv\_profiler/profile.py}, add:

\begin{itemize}
\tightlist
\item
  \texttt{is\_missing(value)}
\item
  \texttt{try\_float(value)}
\item
  \texttt{infer\_type(values)}
\end{itemize}

\textbf{Checkpoint:} you can call these from a Python REPL and they
behave correctly.

\begin{center}\rule{0.5\linewidth}{0.5pt}\end{center}

\subsection{Solution --- helpers
(example)}\label{solution-helpers-example}

\begin{Shaded}
\begin{Highlighting}[]
\NormalTok{MISSING }\OperatorTok{=}\NormalTok{ \{}\StringTok{""}\NormalTok{, }\StringTok{"na"}\NormalTok{, }\StringTok{"n/a"}\NormalTok{, }\StringTok{"null"}\NormalTok{, }\StringTok{"none"}\NormalTok{, }\StringTok{"nan"}\NormalTok{\}}

\KeywordTok{def}\NormalTok{ is\_missing(value: }\BuiltInTok{str} \OperatorTok{|} \VariableTok{None}\NormalTok{) }\OperatorTok{{-}\textgreater{}} \BuiltInTok{bool}\NormalTok{:}
    \ControlFlowTok{if}\NormalTok{ value }\KeywordTok{is} \VariableTok{None}\NormalTok{:}
        \ControlFlowTok{return} \VariableTok{True}
    \ControlFlowTok{return}\NormalTok{ value.strip().casefold() }\KeywordTok{in}\NormalTok{ MISSING}

\KeywordTok{def}\NormalTok{ try\_float(value: }\BuiltInTok{str}\NormalTok{) }\OperatorTok{{-}\textgreater{}} \BuiltInTok{float} \OperatorTok{|} \VariableTok{None}\NormalTok{:}
    \ControlFlowTok{try}\NormalTok{:}
        \ControlFlowTok{return} \BuiltInTok{float}\NormalTok{(value)}
    \ControlFlowTok{except} \PreprocessorTok{ValueError}\NormalTok{:}
        \ControlFlowTok{return} \VariableTok{None}

\KeywordTok{def}\NormalTok{ infer\_type(values: }\BuiltInTok{list}\NormalTok{[}\BuiltInTok{str}\NormalTok{]) }\OperatorTok{{-}\textgreater{}} \BuiltInTok{str}\NormalTok{:}
\NormalTok{    usable }\OperatorTok{=}\NormalTok{ [v }\ControlFlowTok{for}\NormalTok{ v }\KeywordTok{in}\NormalTok{ values }\ControlFlowTok{if} \KeywordTok{not}\NormalTok{ is\_missing(v)]}
    \ControlFlowTok{if} \KeywordTok{not}\NormalTok{ usable:}
        \ControlFlowTok{return} \StringTok{"text"}
    \ControlFlowTok{for}\NormalTok{ v }\KeywordTok{in}\NormalTok{ usable:}
        \ControlFlowTok{if}\NormalTok{ try\_float(v) }\KeywordTok{is} \VariableTok{None}\NormalTok{:}
            \ControlFlowTok{return} \StringTok{"text"}
    \ControlFlowTok{return} \StringTok{"number"}
\end{Highlighting}
\end{Shaded}

\begin{center}\rule{0.5\linewidth}{0.5pt}\end{center}

\subsection{Task 2 --- Extract column values (10
minutes)}\label{task-2-extract-column-values-10-minutes}

Add a helper:

\begin{Shaded}
\begin{Highlighting}[]
\KeywordTok{def}\NormalTok{ column\_values(rows: }\BuiltInTok{list}\NormalTok{[}\BuiltInTok{dict}\NormalTok{[}\BuiltInTok{str}\NormalTok{, }\BuiltInTok{str}\NormalTok{]], col: }\BuiltInTok{str}\NormalTok{) }\OperatorTok{{-}\textgreater{}} \BuiltInTok{list}\NormalTok{[}\BuiltInTok{str}\NormalTok{]:}
\NormalTok{    ...}
\end{Highlighting}
\end{Shaded}

Rules:

\begin{itemize}
\tightlist
\item
  return one value per row (use \texttt{row.get(col,\ "")})
\item
  keep as strings (parsing happens later)
\end{itemize}

\begin{center}\rule{0.5\linewidth}{0.5pt}\end{center}

\subsection{\texorpdfstring{Solution ---
\texttt{column\_values}}{Solution --- column\_values}}\label{solution-column_values}

\begin{Shaded}
\begin{Highlighting}[]
\KeywordTok{def}\NormalTok{ column\_values(rows: }\BuiltInTok{list}\NormalTok{[}\BuiltInTok{dict}\NormalTok{[}\BuiltInTok{str}\NormalTok{, }\BuiltInTok{str}\NormalTok{]], col: }\BuiltInTok{str}\NormalTok{) }\OperatorTok{{-}\textgreater{}} \BuiltInTok{list}\NormalTok{[}\BuiltInTok{str}\NormalTok{]:}
    \ControlFlowTok{return}\NormalTok{ [row.get(col, }\StringTok{""}\NormalTok{) }\ControlFlowTok{for}\NormalTok{ row }\KeywordTok{in}\NormalTok{ rows]}
\end{Highlighting}
\end{Shaded}

\begin{center}\rule{0.5\linewidth}{0.5pt}\end{center}

\subsection{Task 3 --- Numeric stats function (15
minutes)}\label{task-3-numeric-stats-function-15-minutes}

Implement:

\begin{Shaded}
\begin{Highlighting}[]
\KeywordTok{def}\NormalTok{ numeric\_stats(values: }\BuiltInTok{list}\NormalTok{[}\BuiltInTok{str}\NormalTok{]) }\OperatorTok{{-}\textgreater{}} \BuiltInTok{dict}\NormalTok{:}
    \CommentTok{"""Compute stats for numeric column values (strings)."""}
\NormalTok{    ...}
\end{Highlighting}
\end{Shaded}

Requirements:

\begin{itemize}
\tightlist
\item
  ignore missing values
\item
  parse remaining values as floats
\item
  compute: \texttt{count}, \texttt{missing}, \texttt{unique},
  \texttt{min}, \texttt{max}, \texttt{mean}
\end{itemize}

\begin{center}\rule{0.5\linewidth}{0.5pt}\end{center}

\subsection{Hint --- numeric stats
strategy}\label{hint-numeric-stats-strategy}

\begin{enumerate}
\def\labelenumi{\arabic{enumi}.}
\tightlist
\item
  \texttt{usable\ =\ {[}v\ for\ v\ in\ values\ if\ not\ is\_missing(v){]}}
\item
  \texttt{nums\ =\ {[}try\_float(v)\ for\ v\ in\ usable{]}}
\item
  if any \texttt{None} → treat as text elsewhere (don't call this
  function)
\item
  compute stats:

  \begin{itemize}
  \tightlist
  \item
    \texttt{count\ =\ len(nums)}
  \item
    \texttt{unique\ =\ len(set(nums))}
  \item
    \texttt{min(nums)}, \texttt{max(nums)}, \texttt{sum(nums)/count}
  \end{itemize}
\end{enumerate}

\begin{center}\rule{0.5\linewidth}{0.5pt}\end{center}

\subsection{\texorpdfstring{Solution --- \texttt{numeric\_stats}
(example)}{Solution --- numeric\_stats (example)}}\label{solution-numeric_stats-example}

\begin{Shaded}
\begin{Highlighting}[]
\KeywordTok{def}\NormalTok{ numeric\_stats(values: }\BuiltInTok{list}\NormalTok{[}\BuiltInTok{str}\NormalTok{]) }\OperatorTok{{-}\textgreater{}} \BuiltInTok{dict}\NormalTok{:}
\NormalTok{    usable }\OperatorTok{=}\NormalTok{ [v }\ControlFlowTok{for}\NormalTok{ v }\KeywordTok{in}\NormalTok{ values }\ControlFlowTok{if} \KeywordTok{not}\NormalTok{ is\_missing(v)]}
\NormalTok{    missing }\OperatorTok{=} \BuiltInTok{len}\NormalTok{(values) }\OperatorTok{{-}} \BuiltInTok{len}\NormalTok{(usable)}
\NormalTok{    nums: }\BuiltInTok{list}\NormalTok{[}\BuiltInTok{float}\NormalTok{] }\OperatorTok{=}\NormalTok{ []}
    \ControlFlowTok{for}\NormalTok{ v }\KeywordTok{in}\NormalTok{ usable:}
\NormalTok{        x }\OperatorTok{=}\NormalTok{ try\_float(v)}
        \ControlFlowTok{if}\NormalTok{ x }\KeywordTok{is} \VariableTok{None}\NormalTok{:}
            \ControlFlowTok{raise} \PreprocessorTok{ValueError}\NormalTok{(}\SpecialStringTok{f"Non{-}numeric value found: }\SpecialCharTok{\{}\NormalTok{v}\SpecialCharTok{!r\}}\SpecialStringTok{"}\NormalTok{)}
\NormalTok{        nums.append(x)}

\NormalTok{    count }\OperatorTok{=} \BuiltInTok{len}\NormalTok{(nums)}
\NormalTok{    unique }\OperatorTok{=} \BuiltInTok{len}\NormalTok{(}\BuiltInTok{set}\NormalTok{(nums))}
    \ControlFlowTok{return}\NormalTok{ \{}
        \StringTok{"count"}\NormalTok{: count,}
        \StringTok{"missing"}\NormalTok{: missing,}
        \StringTok{"unique"}\NormalTok{: unique,}
        \StringTok{"min"}\NormalTok{: }\BuiltInTok{min}\NormalTok{(nums) }\ControlFlowTok{if}\NormalTok{ nums }\ControlFlowTok{else} \VariableTok{None}\NormalTok{,}
        \StringTok{"max"}\NormalTok{: }\BuiltInTok{max}\NormalTok{(nums) }\ControlFlowTok{if}\NormalTok{ nums }\ControlFlowTok{else} \VariableTok{None}\NormalTok{,}
        \StringTok{"mean"}\NormalTok{: (}\BuiltInTok{sum}\NormalTok{(nums) }\OperatorTok{/}\NormalTok{ count) }\ControlFlowTok{if}\NormalTok{ count }\ControlFlowTok{else} \VariableTok{None}\NormalTok{,}
\NormalTok{    \}}
\end{Highlighting}
\end{Shaded}

\marginnote{\begin{footnotesize}

We raise an error if something is non-numeric. That makes bugs loud.

\end{footnotesize}}

\begin{center}\rule{0.5\linewidth}{0.5pt}\end{center}

\subsection{Task 4 --- Text stats function (15
minutes)}\label{task-4-text-stats-function-15-minutes}

Implement:

\begin{Shaded}
\begin{Highlighting}[]
\KeywordTok{def}\NormalTok{ text\_stats(values: }\BuiltInTok{list}\NormalTok{[}\BuiltInTok{str}\NormalTok{], top\_k: }\BuiltInTok{int} \OperatorTok{=} \DecValTok{5}\NormalTok{) }\OperatorTok{{-}\textgreater{}} \BuiltInTok{dict}\NormalTok{:}
\NormalTok{    ...}
\end{Highlighting}
\end{Shaded}

Requirements:

\begin{itemize}
\tightlist
\item
  ignore missing values
\item
  compute: \texttt{count}, \texttt{missing}, \texttt{unique}
\item
  compute \texttt{top}: top\_k most common values with counts
\end{itemize}

\begin{center}\rule{0.5\linewidth}{0.5pt}\end{center}

\subsection{Hint --- counting text
values}\label{hint-counting-text-values}

You can implement counts with a dict:

\begin{Shaded}
\begin{Highlighting}[]
\NormalTok{counts: }\BuiltInTok{dict}\NormalTok{[}\BuiltInTok{str}\NormalTok{, }\BuiltInTok{int}\NormalTok{] }\OperatorTok{=}\NormalTok{ \{\}}
\ControlFlowTok{for}\NormalTok{ v }\KeywordTok{in}\NormalTok{ usable:}
\NormalTok{    counts[v] }\OperatorTok{=}\NormalTok{ counts.get(v, }\DecValTok{0}\NormalTok{) }\OperatorTok{+} \DecValTok{1}
\end{Highlighting}
\end{Shaded}

Then sort by count descending:

\begin{Shaded}
\begin{Highlighting}[]
\NormalTok{top }\OperatorTok{=} \BuiltInTok{sorted}\NormalTok{(counts.items(), key}\OperatorTok{=}\KeywordTok{lambda}\NormalTok{ kv: kv[}\DecValTok{1}\NormalTok{], reverse}\OperatorTok{=}\VariableTok{True}\NormalTok{)[:top\_k]}
\end{Highlighting}
\end{Shaded}

\begin{center}\rule{0.5\linewidth}{0.5pt}\end{center}

\subsection{\texorpdfstring{Solution --- \texttt{text\_stats}
(example)}{Solution --- text\_stats (example)}}\label{solution-text_stats-example}

\begin{Shaded}
\begin{Highlighting}[]
\KeywordTok{def}\NormalTok{ text\_stats(values: }\BuiltInTok{list}\NormalTok{[}\BuiltInTok{str}\NormalTok{], top\_k: }\BuiltInTok{int} \OperatorTok{=} \DecValTok{5}\NormalTok{) }\OperatorTok{{-}\textgreater{}} \BuiltInTok{dict}\NormalTok{:}
\NormalTok{    usable }\OperatorTok{=}\NormalTok{ [v }\ControlFlowTok{for}\NormalTok{ v }\KeywordTok{in}\NormalTok{ values }\ControlFlowTok{if} \KeywordTok{not}\NormalTok{ is\_missing(v)]}
\NormalTok{    missing }\OperatorTok{=} \BuiltInTok{len}\NormalTok{(values) }\OperatorTok{{-}} \BuiltInTok{len}\NormalTok{(usable)}

\NormalTok{    counts: }\BuiltInTok{dict}\NormalTok{[}\BuiltInTok{str}\NormalTok{, }\BuiltInTok{int}\NormalTok{] }\OperatorTok{=}\NormalTok{ \{\}}
    \ControlFlowTok{for}\NormalTok{ v }\KeywordTok{in}\NormalTok{ usable:}
\NormalTok{        counts[v] }\OperatorTok{=}\NormalTok{ counts.get(v, }\DecValTok{0}\NormalTok{) }\OperatorTok{+} \DecValTok{1}

\NormalTok{    top\_items }\OperatorTok{=} \BuiltInTok{sorted}\NormalTok{(counts.items(), key}\OperatorTok{=}\KeywordTok{lambda}\NormalTok{ kv: kv[}\DecValTok{1}\NormalTok{], reverse}\OperatorTok{=}\VariableTok{True}\NormalTok{)[:top\_k]}
\NormalTok{    top }\OperatorTok{=}\NormalTok{ [\{}\StringTok{"value"}\NormalTok{: v, }\StringTok{"count"}\NormalTok{: c\} }\ControlFlowTok{for}\NormalTok{ v, c }\KeywordTok{in}\NormalTok{ top\_items]}

    \ControlFlowTok{return}\NormalTok{ \{}
        \StringTok{"count"}\NormalTok{: }\BuiltInTok{len}\NormalTok{(usable),}
        \StringTok{"missing"}\NormalTok{: missing,}
        \StringTok{"unique"}\NormalTok{: }\BuiltInTok{len}\NormalTok{(counts),}
        \StringTok{"top"}\NormalTok{: top,}
\NormalTok{    \}}
\end{Highlighting}
\end{Shaded}

\begin{center}\rule{0.5\linewidth}{0.5pt}\end{center}

\subsection{\texorpdfstring{Task 5 --- Upgrade \texttt{basic\_profile}
(20
minutes)}{Task 5 --- Upgrade basic\_profile (20 minutes)}}\label{task-5-upgrade-basic_profile-20-minutes}

Update \texttt{basic\_profile(rows)} so it returns:

\begin{itemize}
\tightlist
\item
  \texttt{source} (path optional for now)
\item
  \texttt{summary}: rows, columns
\item
  \texttt{columns}: a dict keyed by column name with:

  \begin{itemize}
  \tightlist
  \item
    \texttt{type} (\texttt{number}/\texttt{text})
  \item
    stats from \texttt{numeric\_stats} or \texttt{text\_stats}
  \end{itemize}
\end{itemize}

\textbf{Checkpoint:}
\texttt{report{[}"columns"{]}{[}"age"{]}{[}"type"{]}\ ==\ "number"} (for
your sample data).

\begin{center}\rule{0.5\linewidth}{0.5pt}\end{center}

\subsection{\texorpdfstring{Solution --- \texttt{basic\_profile}
(example)}{Solution --- basic\_profile (example)}}\label{solution-basic_profile-example}

\begin{Shaded}
\begin{Highlighting}[]
\KeywordTok{def}\NormalTok{ basic\_profile(rows: }\BuiltInTok{list}\NormalTok{[}\BuiltInTok{dict}\NormalTok{[}\BuiltInTok{str}\NormalTok{, }\BuiltInTok{str}\NormalTok{]]) }\OperatorTok{{-}\textgreater{}} \BuiltInTok{dict}\NormalTok{:}
\NormalTok{    cols }\OperatorTok{=}\NormalTok{ get\_columns(rows)}
\NormalTok{    report }\OperatorTok{=}\NormalTok{ \{}
        \StringTok{"summary"}\NormalTok{: \{}
            \StringTok{"rows"}\NormalTok{: }\BuiltInTok{len}\NormalTok{(rows),}
            \StringTok{"columns"}\NormalTok{: }\BuiltInTok{len}\NormalTok{(cols),}
            \StringTok{"column\_names"}\NormalTok{: cols,}
\NormalTok{        \},}
        \StringTok{"columns"}\NormalTok{: \{\},}
\NormalTok{    \}}

    \ControlFlowTok{for}\NormalTok{ col }\KeywordTok{in}\NormalTok{ cols:}
\NormalTok{        values }\OperatorTok{=}\NormalTok{ column\_values(rows, col)}
\NormalTok{        typ }\OperatorTok{=}\NormalTok{ infer\_type(values)}

        \ControlFlowTok{if}\NormalTok{ typ }\OperatorTok{==} \StringTok{"number"}\NormalTok{:}
\NormalTok{            stats }\OperatorTok{=}\NormalTok{ numeric\_stats(values)}
        \ControlFlowTok{else}\NormalTok{:}
\NormalTok{            stats }\OperatorTok{=}\NormalTok{ text\_stats(values)}

\NormalTok{        report[}\StringTok{"columns"}\NormalTok{][col] }\OperatorTok{=}\NormalTok{ \{}\StringTok{"type"}\NormalTok{: typ, }\OperatorTok{**}\NormalTok{stats\}}

    \ControlFlowTok{return}\NormalTok{ report}
\end{Highlighting}
\end{Shaded}

\begin{center}\rule{0.5\linewidth}{0.5pt}\end{center}

\subsection{\texorpdfstring{Task 6 --- Upgrade \texttt{write\_markdown}
(25
minutes)}{Task 6 --- Upgrade write\_markdown (25 minutes)}}\label{task-6-upgrade-write_markdown-25-minutes}

In \texttt{src/csv\_profiler/render.py}, update
\texttt{write\_markdown(report,\ path)} to include:

\begin{enumerate}
\def\labelenumi{\arabic{enumi}.}
\tightlist
\item
  Header block (\texttt{md\_header(...)})
\item
  Summary bullets:

  \begin{itemize}
  \tightlist
  \item
    rows, columns
  \end{itemize}
\item
  A table: one row per column (type, missing \%, unique)
\item
  Per-column details section:

  \begin{itemize}
  \tightlist
  \item
    For numeric: min/max/mean
  \item
    For text: top values list
  \end{itemize}
\end{enumerate}

\begin{center}\rule{0.5\linewidth}{0.5pt}\end{center}

\subsection{Hint --- computing missing
percentage}\label{hint-computing-missing-percentage}

\begin{Shaded}
\begin{Highlighting}[]
\NormalTok{rows }\OperatorTok{=}\NormalTok{ report[}\StringTok{"summary"}\NormalTok{][}\StringTok{"rows"}\NormalTok{]}
\NormalTok{missing }\OperatorTok{=}\NormalTok{ col\_report[}\StringTok{"missing"}\NormalTok{]}
\NormalTok{missing\_pct }\OperatorTok{=}\NormalTok{ (missing }\OperatorTok{/}\NormalTok{ rows) }\ControlFlowTok{if}\NormalTok{ rows }\ControlFlowTok{else} \FloatTok{0.0}
\end{Highlighting}
\end{Shaded}

\begin{center}\rule{0.5\linewidth}{0.5pt}\end{center}

\subsection{\texorpdfstring{Solution --- \texttt{write\_markdown}
(example,
simple)}{Solution --- write\_markdown (example, simple)}}\label{solution-write_markdown-example-simple}

\begin{Shaded}
\begin{Highlighting}[]
\ImportTok{from}\NormalTok{ pathlib }\ImportTok{import}\NormalTok{ Path}

\KeywordTok{def}\NormalTok{ write\_markdown(report: }\BuiltInTok{dict}\NormalTok{, path: }\BuiltInTok{str} \OperatorTok{|}\NormalTok{ Path) }\OperatorTok{{-}\textgreater{}} \VariableTok{None}\NormalTok{:}
\NormalTok{    path }\OperatorTok{=}\NormalTok{ Path(path)}
\NormalTok{    path.parent.mkdir(parents}\OperatorTok{=}\VariableTok{True}\NormalTok{, exist\_ok}\OperatorTok{=}\VariableTok{True}\NormalTok{)}

\NormalTok{    rows }\OperatorTok{=}\NormalTok{ report[}\StringTok{"summary"}\NormalTok{][}\StringTok{"rows"}\NormalTok{]}

\NormalTok{    lines: }\BuiltInTok{list}\NormalTok{[}\BuiltInTok{str}\NormalTok{] }\OperatorTok{=}\NormalTok{ []}
\NormalTok{    lines.extend(md\_header(}\StringTok{"data/sample.csv"}\NormalTok{))}

\NormalTok{    lines.append(}\StringTok{"\#\# Summary"}\NormalTok{)}
\NormalTok{    lines.append(}\SpecialStringTok{f"{-} Rows: }\SpecialCharTok{\{}\NormalTok{rows}\SpecialCharTok{:,\}}\SpecialStringTok{"}\NormalTok{)}
\NormalTok{    lines.append(}\SpecialStringTok{f"{-} Columns: }\SpecialCharTok{\{}\NormalTok{report[}\StringTok{\textquotesingle{}summary\textquotesingle{}}\NormalTok{][}\StringTok{\textquotesingle{}columns\textquotesingle{}}\NormalTok{]}\SpecialCharTok{:,\}}\SpecialStringTok{"}\NormalTok{)}
\NormalTok{    lines.append(}\StringTok{""}\NormalTok{)}

\NormalTok{    lines.append(}\StringTok{"\#\# Columns (table)"}\NormalTok{)}
\NormalTok{    lines.extend(md\_table\_header())}

    \ControlFlowTok{for}\NormalTok{ name, col }\KeywordTok{in}\NormalTok{ report[}\StringTok{"columns"}\NormalTok{].items():}
\NormalTok{        missing\_pct }\OperatorTok{=}\NormalTok{ (col[}\StringTok{"missing"}\NormalTok{] }\OperatorTok{/}\NormalTok{ rows) }\ControlFlowTok{if}\NormalTok{ rows }\ControlFlowTok{else} \FloatTok{0.0}
\NormalTok{        lines.append(md\_col\_row(name, col[}\StringTok{"type"}\NormalTok{], col[}\StringTok{"missing"}\NormalTok{], missing\_pct, col[}\StringTok{"unique"}\NormalTok{]))}

\NormalTok{    lines.append(}\StringTok{""}\NormalTok{)}
\NormalTok{    lines.append(}\StringTok{"\#\# Column details"}\NormalTok{)}

    \ControlFlowTok{for}\NormalTok{ name, col }\KeywordTok{in}\NormalTok{ report[}\StringTok{"columns"}\NormalTok{].items():}
\NormalTok{        lines.append(}\SpecialStringTok{f"\#\#\# \textasciigrave{}}\SpecialCharTok{\{}\NormalTok{name}\SpecialCharTok{\}}\SpecialStringTok{\textasciigrave{} (}\SpecialCharTok{\{}\NormalTok{col[}\StringTok{\textquotesingle{}type\textquotesingle{}}\NormalTok{]}\SpecialCharTok{\}}\SpecialStringTok{)"}\NormalTok{)}

        \ControlFlowTok{if}\NormalTok{ col[}\StringTok{"type"}\NormalTok{] }\OperatorTok{==} \StringTok{"number"}\NormalTok{:}
\NormalTok{            lines.append(}\SpecialStringTok{f"{-} min: }\SpecialCharTok{\{}\NormalTok{col[}\StringTok{\textquotesingle{}min\textquotesingle{}}\NormalTok{]}\SpecialCharTok{\}}\SpecialStringTok{"}\NormalTok{)}
\NormalTok{            lines.append(}\SpecialStringTok{f"{-} max: }\SpecialCharTok{\{}\NormalTok{col[}\StringTok{\textquotesingle{}max\textquotesingle{}}\NormalTok{]}\SpecialCharTok{\}}\SpecialStringTok{"}\NormalTok{)}
\NormalTok{            lines.append(}\SpecialStringTok{f"{-} mean: }\SpecialCharTok{\{}\NormalTok{col[}\StringTok{\textquotesingle{}mean\textquotesingle{}}\NormalTok{]}\SpecialCharTok{\}}\SpecialStringTok{"}\NormalTok{)}
        \ControlFlowTok{else}\NormalTok{:}
\NormalTok{            top }\OperatorTok{=}\NormalTok{ col.get(}\StringTok{"top"}\NormalTok{, [])}
            \ControlFlowTok{if} \KeywordTok{not}\NormalTok{ top:}
\NormalTok{                lines.append(}\StringTok{"{-} (no non{-}missing values)"}\NormalTok{)}
            \ControlFlowTok{else}\NormalTok{:}
\NormalTok{                lines.append(}\StringTok{"{-} top values:"}\NormalTok{)}
                \ControlFlowTok{for}\NormalTok{ item }\KeywordTok{in}\NormalTok{ top:}
\NormalTok{                    lines.append(}\SpecialStringTok{f"  {-} \textasciigrave{}}\SpecialCharTok{\{}\NormalTok{item[}\StringTok{\textquotesingle{}value\textquotesingle{}}\NormalTok{]}\SpecialCharTok{\}}\SpecialStringTok{\textasciigrave{}: }\SpecialCharTok{\{}\NormalTok{item[}\StringTok{\textquotesingle{}count\textquotesingle{}}\NormalTok{]}\SpecialCharTok{\}}\SpecialStringTok{"}\NormalTok{)}

\NormalTok{        lines.append(}\StringTok{""}\NormalTok{)}

\NormalTok{    path.write\_text(}\StringTok{"}\CharTok{\textbackslash{}n}\StringTok{"}\NormalTok{.join(lines) }\OperatorTok{+} \StringTok{"}\CharTok{\textbackslash{}n}\StringTok{"}\NormalTok{, encoding}\OperatorTok{=}\StringTok{"utf{-}8"}\NormalTok{)}
\end{Highlighting}
\end{Shaded}

\begin{tcolorbox}[enhanced jigsaw, breakable, opacitybacktitle=0.6, toptitle=1mm, bottomrule=.15mm, arc=.35mm, toprule=.15mm, bottomtitle=1mm, coltitle=black, titlerule=0mm, title=\textcolor{quarto-callout-tip-color}{\faLightbulb}\hspace{0.5em}{Tip}, rightrule=.15mm, leftrule=.75mm, colback=white, left=2mm, opacityback=0, colframe=quarto-callout-tip-color-frame, colbacktitle=quarto-callout-tip-color!10!white]

Keep it ``simple but correct'' today. We'll polish the report formatting
later.

\end{tcolorbox}

\begin{center}\rule{0.5\linewidth}{0.5pt}\end{center}

\subsection{Task 7 --- Run end-to-end (10
minutes)}\label{task-7-run-end-to-end-10-minutes}

Run:

\begin{itemize}
\tightlist
\item
  Unix/macOS:
\end{itemize}

\begin{Shaded}
\begin{Highlighting}[]
\VariableTok{PYTHONPATH}\OperatorTok{=}\NormalTok{src }\ExtensionTok{uv}\NormalTok{ run python main.py}
\end{Highlighting}
\end{Shaded}

\begin{itemize}
\tightlist
\item
  Windows PowerShell:
\end{itemize}

\begin{Shaded}
\begin{Highlighting}[]
\VariableTok{$env}\OperatorTok{:}\VariableTok{PYTHONPATH}\OperatorTok{=}\StringTok{"src"}
\NormalTok{uv run python main}\OperatorTok{.}\FunctionTok{py}
\end{Highlighting}
\end{Shaded}

\textbf{Checkpoint:} - \texttt{outputs/report.json} has types and stats
- \texttt{outputs/report.md} has a table and details

\begin{center}\rule{0.5\linewidth}{0.5pt}\end{center}

\subsection{Debug playbook (when it
fails)}\label{debug-playbook-when-it-fails}

\begin{enumerate}
\def\labelenumi{\arabic{enumi}.}
\tightlist
\item
  Read the traceback \textbf{top to bottom}
\item
  Find the first line that points to \textbf{your code}
\item
  Print intermediate values:
\end{enumerate}

\begin{Shaded}
\begin{Highlighting}[]
\BuiltInTok{print}\NormalTok{(}\StringTok{"DEBUG values:"}\NormalTok{, values[:}\DecValTok{5}\NormalTok{])}
\end{Highlighting}
\end{Shaded}

\begin{enumerate}
\def\labelenumi{\arabic{enumi}.}
\setcounter{enumi}{3}
\tightlist
\item
  Confirm assumptions:
\end{enumerate}

\begin{itemize}
\tightlist
\item
  are you reading the file you think you are?
\item
  are there missing keys?
\item
  are strings like \texttt{"\ "} being treated as missing?
\end{itemize}

\begin{center}\rule{0.5\linewidth}{0.5pt}\end{center}

\subsection{Stretch (if you finish
early)}\label{stretch-if-you-finish-early}

Pick \textbf{one}:

\begin{itemize}
\tightlist
\item
  Add \texttt{median} (hint: sort and pick middle)
\item
  Add a ``mostly numeric'' rule (e.g., ≥ 90\% parse as float)
\item
  Add a \texttt{-\/-input} / \texttt{-\/-output} argument using
  \texttt{argparse} \emph{(Typer comes tomorrow)}
\end{itemize}

\begin{center}\rule{0.5\linewidth}{0.5pt}\end{center}

\subsection{Exit Ticket}\label{exit-ticket}

In 1--2 sentences:

\textbf{What made your profiler ``better'' today compared to Day 1?}

\begin{center}\rule{0.5\linewidth}{0.5pt}\end{center}

\subsection{What to do after class (Day 2
assignment)}\label{what-to-do-after-class-day-2-assignment}

\textbf{Due:} before Day 3 starts (Tue, 16 Dec 2025)

\begin{enumerate}
\def\labelenumi{\arabic{enumi}.}
\tightlist
\item
  Add at least \textbf{2 new columns} to \texttt{data/sample.csv} (one
  numeric, one text)
\item
  Rerun the profiler and check that:

  \begin{itemize}
  \tightlist
  \item
    types are correct
  \item
    stats update correctly
  \end{itemize}
\item
  Improve your Markdown:

  \begin{itemize}
  \tightlist
  \item
    format numeric values with \texttt{:.2f} where relevant
  \item
    show missing percentage in the table
  \end{itemize}
\end{enumerate}

\textbf{Deliverable:} a zip or folder with your updated
\texttt{csv-profiler/}.

\section{Thank You!}\label{thank-you}




\end{document}
