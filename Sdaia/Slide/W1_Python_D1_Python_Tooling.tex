% Options for packages loaded elsewhere
% Options for packages loaded elsewhere
\PassOptionsToPackage{unicode}{hyperref}
\PassOptionsToPackage{hyphens}{url}
\PassOptionsToPackage{dvipsnames,svgnames,x11names}{xcolor}
%
\documentclass[
  letterpaper,
  DIV=11,
  numbers=noendperiod,
  oneside]{scrartcl}
\usepackage{xcolor}
\usepackage[left=1in,marginparwidth=2.0666666666667in,textwidth=4.1333333333333in,marginparsep=0.3in]{geometry}
\usepackage{amsmath,amssymb}
\setcounter{secnumdepth}{-\maxdimen} % remove section numbering
\usepackage{iftex}
\ifPDFTeX
  \usepackage[T1]{fontenc}
  \usepackage[utf8]{inputenc}
  \usepackage{textcomp} % provide euro and other symbols
\else % if luatex or xetex
  \usepackage{unicode-math} % this also loads fontspec
  \defaultfontfeatures{Scale=MatchLowercase}
  \defaultfontfeatures[\rmfamily]{Ligatures=TeX,Scale=1}
\fi
\usepackage{lmodern}
\ifPDFTeX\else
  % xetex/luatex font selection
\fi
% Use upquote if available, for straight quotes in verbatim environments
\IfFileExists{upquote.sty}{\usepackage{upquote}}{}
\IfFileExists{microtype.sty}{% use microtype if available
  \usepackage[]{microtype}
  \UseMicrotypeSet[protrusion]{basicmath} % disable protrusion for tt fonts
}{}
\makeatletter
\@ifundefined{KOMAClassName}{% if non-KOMA class
  \IfFileExists{parskip.sty}{%
    \usepackage{parskip}
  }{% else
    \setlength{\parindent}{0pt}
    \setlength{\parskip}{6pt plus 2pt minus 1pt}}
}{% if KOMA class
  \KOMAoptions{parskip=half}}
\makeatother
% Make \paragraph and \subparagraph free-standing
\makeatletter
\ifx\paragraph\undefined\else
  \let\oldparagraph\paragraph
  \renewcommand{\paragraph}{
    \@ifstar
      \xxxParagraphStar
      \xxxParagraphNoStar
  }
  \newcommand{\xxxParagraphStar}[1]{\oldparagraph*{#1}\mbox{}}
  \newcommand{\xxxParagraphNoStar}[1]{\oldparagraph{#1}\mbox{}}
\fi
\ifx\subparagraph\undefined\else
  \let\oldsubparagraph\subparagraph
  \renewcommand{\subparagraph}{
    \@ifstar
      \xxxSubParagraphStar
      \xxxSubParagraphNoStar
  }
  \newcommand{\xxxSubParagraphStar}[1]{\oldsubparagraph*{#1}\mbox{}}
  \newcommand{\xxxSubParagraphNoStar}[1]{\oldsubparagraph{#1}\mbox{}}
\fi
\makeatother

\usepackage{color}
\usepackage{fancyvrb}
\newcommand{\VerbBar}{|}
\newcommand{\VERB}{\Verb[commandchars=\\\{\}]}
\DefineVerbatimEnvironment{Highlighting}{Verbatim}{commandchars=\\\{\}}
% Add ',fontsize=\small' for more characters per line
\usepackage{framed}
\definecolor{shadecolor}{RGB}{241,243,245}
\newenvironment{Shaded}{\begin{snugshade}}{\end{snugshade}}
\newcommand{\AlertTok}[1]{\textcolor[rgb]{0.68,0.00,0.00}{#1}}
\newcommand{\AnnotationTok}[1]{\textcolor[rgb]{0.37,0.37,0.37}{#1}}
\newcommand{\AttributeTok}[1]{\textcolor[rgb]{0.40,0.45,0.13}{#1}}
\newcommand{\BaseNTok}[1]{\textcolor[rgb]{0.68,0.00,0.00}{#1}}
\newcommand{\BuiltInTok}[1]{\textcolor[rgb]{0.00,0.23,0.31}{#1}}
\newcommand{\CharTok}[1]{\textcolor[rgb]{0.13,0.47,0.30}{#1}}
\newcommand{\CommentTok}[1]{\textcolor[rgb]{0.37,0.37,0.37}{#1}}
\newcommand{\CommentVarTok}[1]{\textcolor[rgb]{0.37,0.37,0.37}{\textit{#1}}}
\newcommand{\ConstantTok}[1]{\textcolor[rgb]{0.56,0.35,0.01}{#1}}
\newcommand{\ControlFlowTok}[1]{\textcolor[rgb]{0.00,0.23,0.31}{\textbf{#1}}}
\newcommand{\DataTypeTok}[1]{\textcolor[rgb]{0.68,0.00,0.00}{#1}}
\newcommand{\DecValTok}[1]{\textcolor[rgb]{0.68,0.00,0.00}{#1}}
\newcommand{\DocumentationTok}[1]{\textcolor[rgb]{0.37,0.37,0.37}{\textit{#1}}}
\newcommand{\ErrorTok}[1]{\textcolor[rgb]{0.68,0.00,0.00}{#1}}
\newcommand{\ExtensionTok}[1]{\textcolor[rgb]{0.00,0.23,0.31}{#1}}
\newcommand{\FloatTok}[1]{\textcolor[rgb]{0.68,0.00,0.00}{#1}}
\newcommand{\FunctionTok}[1]{\textcolor[rgb]{0.28,0.35,0.67}{#1}}
\newcommand{\ImportTok}[1]{\textcolor[rgb]{0.00,0.46,0.62}{#1}}
\newcommand{\InformationTok}[1]{\textcolor[rgb]{0.37,0.37,0.37}{#1}}
\newcommand{\KeywordTok}[1]{\textcolor[rgb]{0.00,0.23,0.31}{\textbf{#1}}}
\newcommand{\NormalTok}[1]{\textcolor[rgb]{0.00,0.23,0.31}{#1}}
\newcommand{\OperatorTok}[1]{\textcolor[rgb]{0.37,0.37,0.37}{#1}}
\newcommand{\OtherTok}[1]{\textcolor[rgb]{0.00,0.23,0.31}{#1}}
\newcommand{\PreprocessorTok}[1]{\textcolor[rgb]{0.68,0.00,0.00}{#1}}
\newcommand{\RegionMarkerTok}[1]{\textcolor[rgb]{0.00,0.23,0.31}{#1}}
\newcommand{\SpecialCharTok}[1]{\textcolor[rgb]{0.37,0.37,0.37}{#1}}
\newcommand{\SpecialStringTok}[1]{\textcolor[rgb]{0.13,0.47,0.30}{#1}}
\newcommand{\StringTok}[1]{\textcolor[rgb]{0.13,0.47,0.30}{#1}}
\newcommand{\VariableTok}[1]{\textcolor[rgb]{0.07,0.07,0.07}{#1}}
\newcommand{\VerbatimStringTok}[1]{\textcolor[rgb]{0.13,0.47,0.30}{#1}}
\newcommand{\WarningTok}[1]{\textcolor[rgb]{0.37,0.37,0.37}{\textit{#1}}}

\usepackage{longtable,booktabs,array}
\usepackage{calc} % for calculating minipage widths
% Correct order of tables after \paragraph or \subparagraph
\usepackage{etoolbox}
\makeatletter
\patchcmd\longtable{\par}{\if@noskipsec\mbox{}\fi\par}{}{}
\makeatother
% Allow footnotes in longtable head/foot
\IfFileExists{footnotehyper.sty}{\usepackage{footnotehyper}}{\usepackage{footnote}}
\makesavenoteenv{longtable}
\usepackage{graphicx}
\makeatletter
\newsavebox\pandoc@box
\newcommand*\pandocbounded[1]{% scales image to fit in text height/width
  \sbox\pandoc@box{#1}%
  \Gscale@div\@tempa{\textheight}{\dimexpr\ht\pandoc@box+\dp\pandoc@box\relax}%
  \Gscale@div\@tempb{\linewidth}{\wd\pandoc@box}%
  \ifdim\@tempb\p@<\@tempa\p@\let\@tempa\@tempb\fi% select the smaller of both
  \ifdim\@tempa\p@<\p@\scalebox{\@tempa}{\usebox\pandoc@box}%
  \else\usebox{\pandoc@box}%
  \fi%
}
% Set default figure placement to htbp
\def\fps@figure{htbp}
\makeatother





\setlength{\emergencystretch}{3em} % prevent overfull lines

\providecommand{\tightlist}{%
  \setlength{\itemsep}{0pt}\setlength{\parskip}{0pt}}



 


\KOMAoption{captions}{tableheading}
\makeatletter
\@ifpackageloaded{tcolorbox}{}{\usepackage[skins,breakable]{tcolorbox}}
\@ifpackageloaded{fontawesome5}{}{\usepackage{fontawesome5}}
\definecolor{quarto-callout-color}{HTML}{909090}
\definecolor{quarto-callout-note-color}{HTML}{0758E5}
\definecolor{quarto-callout-important-color}{HTML}{CC1914}
\definecolor{quarto-callout-warning-color}{HTML}{EB9113}
\definecolor{quarto-callout-tip-color}{HTML}{00A047}
\definecolor{quarto-callout-caution-color}{HTML}{FC5300}
\definecolor{quarto-callout-color-frame}{HTML}{acacac}
\definecolor{quarto-callout-note-color-frame}{HTML}{4582ec}
\definecolor{quarto-callout-important-color-frame}{HTML}{d9534f}
\definecolor{quarto-callout-warning-color-frame}{HTML}{f0ad4e}
\definecolor{quarto-callout-tip-color-frame}{HTML}{02b875}
\definecolor{quarto-callout-caution-color-frame}{HTML}{fd7e14}
\makeatother
\makeatletter
\@ifpackageloaded{caption}{}{\usepackage{caption}}
\AtBeginDocument{%
\ifdefined\contentsname
  \renewcommand*\contentsname{Table of contents}
\else
  \newcommand\contentsname{Table of contents}
\fi
\ifdefined\listfigurename
  \renewcommand*\listfigurename{List of Figures}
\else
  \newcommand\listfigurename{List of Figures}
\fi
\ifdefined\listtablename
  \renewcommand*\listtablename{List of Tables}
\else
  \newcommand\listtablename{List of Tables}
\fi
\ifdefined\figurename
  \renewcommand*\figurename{Figure}
\else
  \newcommand\figurename{Figure}
\fi
\ifdefined\tablename
  \renewcommand*\tablename{Table}
\else
  \newcommand\tablename{Table}
\fi
}
\@ifpackageloaded{float}{}{\usepackage{float}}
\floatstyle{ruled}
\@ifundefined{c@chapter}{\newfloat{codelisting}{h}{lop}}{\newfloat{codelisting}{h}{lop}[chapter]}
\floatname{codelisting}{Listing}
\newcommand*\listoflistings{\listof{codelisting}{List of Listings}}
\makeatother
\makeatletter
\makeatother
\makeatletter
\@ifpackageloaded{caption}{}{\usepackage{caption}}
\@ifpackageloaded{subcaption}{}{\usepackage{subcaption}}
\makeatother
\makeatletter
\@ifpackageloaded{sidenotes}{}{\usepackage{sidenotes}}
\@ifpackageloaded{marginnote}{}{\usepackage{marginnote}}
\makeatother
\usepackage{bookmark}
\IfFileExists{xurl.sty}{\usepackage{xurl}}{} % add URL line breaks if available
\urlstyle{same}
\hypersetup{
  pdftitle={Python \& Tooling},
  pdfauthor={malfadly@sdaia.gov.sa},
  colorlinks=true,
  linkcolor={blue},
  filecolor={Maroon},
  citecolor={Blue},
  urlcolor={Blue},
  pdfcreator={LaTeX via pandoc}}


\title{Python \& Tooling}
\usepackage{etoolbox}
\makeatletter
\providecommand{\subtitle}[1]{% add subtitle to \maketitle
  \apptocmd{\@title}{\par {\large #1 \par}}{}{}
}
\makeatother
\subtitle{AI Professionals Bootcamp \textbar{} Week 1}
\author{}
\date{2025-12-14}
\begin{document}
\maketitle


\section{Day 1: Python \& Tooling}\label{day-1-python-tooling}

\textbf{Goal:} Set up your environment, use the shell confidently, and
write your first Python scripts that read a CSV and produce a basic
profile.

Bootcamp • SDAIA Academy

\textbf{Say:} This week is ``no GenAI'' (except clarifying questions).
We're building fundamentals.

\textbf{Do:} Confirm everyone can open a terminal + VS Code.

\textbf{Ask:} ``What's harder for you right now: terminal, Python
syntax, or Git?''

\textbf{Timebox:} 3 minutes.

\begin{center}\rule{0.5\linewidth}{0.5pt}\end{center}

\subsection{Today's Flow}\label{todays-flow}

\begin{itemize}
\tightlist
\item
  \textbf{Session 1 (60m):} Setup + Shell essentials + \texttt{uv}
\item
  \emph{Asr Prayer (20m)}
\item
  \textbf{Session 2 (60m):} Values, containers, operators
\item
  \emph{Maghrib Prayer (20m)}
\item
  \textbf{Session 3 (60m):} Control flow + types + files
\item
  \emph{Isha Prayer (20m)}
\item
  \textbf{Hands-on (120m):} weekly project start (CSV Profiler)
\end{itemize}

\begin{center}\rule{0.5\linewidth}{0.5pt}\end{center}

\subsection{Learning Objectives}\label{learning-objectives}

By the end of today, you can:

\begin{itemize}
\tightlist
\item
  Navigate and inspect files using basic shell commands
\item
  Create and use a Python environment with \texttt{uv}
\item
  Write Python scripts with variables, basic types, and control flow
\item
  Read a CSV and compute a \textbf{basic profiling summary}
\item
  Write outputs to \textbf{Markdown} and \textbf{JSON} files
\end{itemize}

\begin{center}\rule{0.5\linewidth}{0.5pt}\end{center}

\subsection{Week 1 outcomes (ship by Thu
11:59pm)}\label{week-1-outcomes-ship-by-thu-1159pm}

You will build a small app with \textbf{two interfaces}:

\begin{itemize}
\tightlist
\item
  \textbf{CLI} (command line) → reads CSV → writes \texttt{report.md}
  and \texttt{report.json}
\item
  \textbf{GUI} with \textbf{Streamlit} → uploads/reads CSV → shows
  profile → export files
\end{itemize}

\marginnote{\begin{footnotesize}

This week is fundamentals: you'll implement the profiling logic
yourself.

\end{footnotesize}}

\begin{center}\rule{0.5\linewidth}{0.5pt}\end{center}

\subsection{Policy: GenAI usage (Week
1)}\label{policy-genai-usage-week-1}

\begin{itemize}
\tightlist
\item
  ✅ Allowed: \textbf{clarifying questions} (definitions, error
  explanations)
\item
  ❌ Not allowed: generating code, writing solutions, or debugging by
  copy-paste
\item
  If unsure: ask the instructor first
\end{itemize}

\begin{tcolorbox}[enhanced jigsaw, coltitle=black, bottomtitle=1mm, arc=.35mm, opacityback=0, colframe=quarto-callout-warning-color-frame, toptitle=1mm, colbacktitle=quarto-callout-warning-color!10!white, breakable, toprule=.15mm, titlerule=0mm, leftrule=.75mm, bottomrule=.15mm, rightrule=.15mm, title=\textcolor{quarto-callout-warning-color}{\faExclamationTriangle}\hspace{0.5em}{Warning}, opacitybacktitle=0.6, left=2mm, colback=white]

This week's point is skill-building. Using GenAI to do the work breaks
the learning loop.

\end{tcolorbox}

\begin{center}\rule{0.5\linewidth}{0.5pt}\end{center}

\subsection{Certificates
(Bootcamp-wide)}\label{certificates-bootcamp-wide}

\begin{itemize}
\tightlist
\item
  \textbf{Certificate of Completion:} final grade \textbf{≥ 70\%} by end
  of the bootcamp
\item
  \textbf{Certificate of Attendance:} if not passing, but \textbf{fewer
  than 4 excused absences}
\end{itemize}

\begin{center}\rule{0.5\linewidth}{0.5pt}\end{center}

\subsection{Bootcamp calendar (so you see the
path)}\label{bootcamp-calendar-so-you-see-the-path}

\begin{itemize}
\tightlist
\item
  Week 01 (14--18 Dec 2025): \textbf{Python \& Tooling}
\item
  Week 02: Data Work (ETL + EDA)
\item
  Week 03: Machine Learning
\item
  Week 04: Deep Learning \& Computer Vision
\item
  Week 05: LLM-based NLP
\item
  Week 06: Building AI Apps
\item
  Week 07: Agentic AI \& Practical MLOps
\item
  Week 08: Capstone Sprint + Job Readiness
\end{itemize}

\begin{center}\rule{0.5\linewidth}{0.5pt}\end{center}

\subsection{Our Week 1 project: ``CSV
Profiler''}\label{our-week-1-project-csv-profiler}

\textbf{Input:} a CSV file

\textbf{Output:}

\begin{itemize}
\tightlist
\item
  \texttt{report.json} → machine-readable profiling stats
\item
  \texttt{report.md} → human-readable report
\end{itemize}

\textbf{Your code} will handle:

\begin{itemize}
\tightlist
\item
  Missing values
\item
  Inferred column types (number / text / mixed)
\item
  Basic stats (count, unique, min/max/mean when numeric)
\end{itemize}

\begin{center}\rule{0.5\linewidth}{0.5pt}\end{center}

\subsection{How we'll work in class}\label{how-well-work-in-class}

\begin{itemize}
\tightlist
\item
  Short chunks of theory
\item
  Micro-exercises (3--8 minutes)
\item
  Checkpoints every \textasciitilde15 minutes
\item
  ``Hands-on'' block = build the project (with help)
\end{itemize}

\begin{tcolorbox}[enhanced jigsaw, coltitle=black, bottomtitle=1mm, arc=.35mm, opacityback=0, colframe=quarto-callout-tip-color-frame, toptitle=1mm, colbacktitle=quarto-callout-tip-color!10!white, breakable, toprule=.15mm, titlerule=0mm, leftrule=.75mm, bottomrule=.15mm, rightrule=.15mm, title=\textcolor{quarto-callout-tip-color}{\faLightbulb}\hspace{0.5em}{Tip}, opacitybacktitle=0.6, left=2mm, colback=white]

If you get stuck: write down the \textbf{exact error}, the
\textbf{command you ran}, and the \textbf{file you edited}.

\end{tcolorbox}

\section{Session 1}\label{session-1}

\subsection{\texorpdfstring{Setup + Shell essentials +
\texttt{uv}}{Setup + Shell essentials + uv}}\label{setup-shell-essentials-uv}

3:00pm--4:00pm

\begin{center}\rule{0.5\linewidth}{0.5pt}\end{center}

\subsection{Session 1 objectives}\label{session-1-objectives}

\begin{itemize}
\tightlist
\item
  Open a terminal and move around the filesystem
\item
  Understand paths: absolute vs relative
\item
  Find your Python and inspect environment variables
\item
  Create a Python env and run a script with \texttt{uv}
\end{itemize}

\begin{center}\rule{0.5\linewidth}{0.5pt}\end{center}

\subsection{Terminal vocabulary}\label{terminal-vocabulary}

\begin{itemize}
\tightlist
\item
  \textbf{Terminal}: the window
\item
  \textbf{Shell}: the program that reads your commands (bash, zsh,
  PowerShell)
\item
  \textbf{Command}: a program you run (\texttt{ls}, \texttt{python},
  \texttt{git})
\item
  \textbf{Working directory}: ``where you are'' right now
\end{itemize}

\begin{center}\rule{0.5\linewidth}{0.5pt}\end{center}

\subsection{IDEs you can use (pick
one)}\label{ides-you-can-use-pick-one}

\begin{itemize}
\tightlist
\item
  \textbf{VS Code} (recommended for this bootcamp)
\item
  \textbf{JupyterLab} (great for exploration)
\item
  \textbf{Google Colab} (only when local setup is blocked)
\end{itemize}

\marginnote{\begin{footnotesize}

Today we'll work mostly with scripts in VS Code.

\end{footnotesize}}

\begin{center}\rule{0.5\linewidth}{0.5pt}\end{center}

\subsection{Navigation: where am I?}\label{navigation-where-am-i}

\textbf{Command}

\begin{itemize}
\tightlist
\item
  \texttt{pwd} → print working directory
\item
  \texttt{ls} (mac/linux) or \texttt{dir} (Windows) → list files
\end{itemize}

\textbf{Try it}

\begin{enumerate}
\def\labelenumi{\arabic{enumi}.}
\tightlist
\item
  Run \texttt{pwd}
\item
  Run \texttt{ls} / \texttt{dir}
\item
  Find your ``Downloads'' or ``Desktop'' folder
\end{enumerate}

\textbf{Do:} Live demo: \texttt{pwd} then \texttt{ls}, explain ``working
directory''.

\textbf{Ask:} ``What folder are you in right now?''

\textbf{Timebox:} 4 minutes.

\begin{center}\rule{0.5\linewidth}{0.5pt}\end{center}

\subsection{Navigation: moving around}\label{navigation-moving-around}

\begin{itemize}
\tightlist
\item
  \texttt{cd\ \textless{}path\textgreater{}} → change directory
\item
  \texttt{cd\ ..} → go up one folder
\item
  \texttt{cd\ .} → current folder (rarely useful)
\item
  \texttt{cd\ \textasciitilde{}} → home folder
\item
  \texttt{cd\ -} → previous folder (super useful)
\end{itemize}

\begin{center}\rule{0.5\linewidth}{0.5pt}\end{center}

\subsection{Paths: absolute vs
relative}\label{paths-absolute-vs-relative}

\textbf{Absolute path} starts from the root.

\begin{itemize}
\tightlist
\item
  mac/linux: \texttt{/Users/\textless{}name\textgreater{}/...}
\item
  Windows:
  \texttt{C:\textbackslash{}Users\textbackslash{}\textless{}name\textgreater{}\textbackslash{}...}
\end{itemize}

\textbf{Relative path} starts from your current folder.

\begin{itemize}
\tightlist
\item
  \texttt{./data/sample.csv} (inside current folder)
\item
  \texttt{../data/sample.csv} (one level up)
\end{itemize}

\begin{center}\rule{0.5\linewidth}{0.5pt}\end{center}

\subsection{Path gotchas (avoid 20 minutes of
pain)}\label{path-gotchas-avoid-20-minutes-of-pain}

\begin{itemize}
\tightlist
\item
  Spaces in folder names can confuse commands → use quotes

  \begin{itemize}
  \tightlist
  \item
    \texttt{cd\ "My\ Files"}
  \end{itemize}
\item
  Case matters on mac/linux (\texttt{Data} ≠ \texttt{data})
\item
  Prefer putting your project in a simple path like
  \texttt{\textasciitilde{}/bootcamp/}
\end{itemize}

\begin{center}\rule{0.5\linewidth}{0.5pt}\end{center}

\subsection{Micro-exercise: ``Path ninja'' (5
minutes)}\label{micro-exercise-path-ninja-5-minutes}

\begin{enumerate}
\def\labelenumi{\arabic{enumi}.}
\tightlist
\item
  \texttt{cd\ \textasciitilde{}}
\item
  Create a new folder called \texttt{bootcamp} \emph{(use your file
  explorer if needed)}
\item
  \texttt{cd\ bootcamp}
\item
  Confirm with \texttt{pwd} and \texttt{ls} / \texttt{dir}
\end{enumerate}

\textbf{Checkpoint:} your terminal shows you are inside
\texttt{bootcamp}.

\begin{center}\rule{0.5\linewidth}{0.5pt}\end{center}

\subsection{Environment variables (why you
care)}\label{environment-variables-why-you-care}

\begin{itemize}
\tightlist
\item
  They are \textbf{settings} for programs
\item
  Most common: \texttt{PATH} (where your shell looks for commands)
\end{itemize}

Try:

\begin{itemize}
\tightlist
\item
  \texttt{echo\ \$PATH} (mac/linux)
\item
  \texttt{echo\ \$env:PATH} (PowerShell)
\end{itemize}

\begin{center}\rule{0.5\linewidth}{0.5pt}\end{center}

\subsection{Finding executables}\label{finding-executables}

\begin{itemize}
\tightlist
\item
  \texttt{which\ python} (mac/linux)
\item
  \texttt{where\ python} (Windows)
\end{itemize}

\textbf{Interpretation:}

\begin{itemize}
\tightlist
\item
  If you see a path inside \texttt{.venv/} → you are in a virtual
  environment
\item
  If you see a system path → you are using system Python
\end{itemize}

\begin{center}\rule{0.5\linewidth}{0.5pt}\end{center}

\subsection{Why virtual environments?}\label{why-virtual-environments}

Different projects need different packages.

\begin{itemize}
\tightlist
\item
  ✅ reproducible installs
\item
  ✅ no ``works on my machine''
\item
  ✅ you can safely delete and recreate
\end{itemize}

\begin{center}\rule{0.5\linewidth}{0.5pt}\end{center}

\subsection{\texorpdfstring{\texttt{uv}: our tool for environments +
installs}{uv: our tool for environments + installs}}\label{uv-our-tool-for-environments-installs}

Today we'll use:

\begin{Shaded}
\begin{Highlighting}[]
\ExtensionTok{uv}\NormalTok{ venv }\AttributeTok{{-}p}\NormalTok{ 3.11}
\ExtensionTok{uv}\NormalTok{ pip install }\OperatorTok{\textless{}}\NormalTok{package}\OperatorTok{\textgreater{}}
\ExtensionTok{uv}\NormalTok{ run }\OperatorTok{\textless{}}\NormalTok{script.py}\OperatorTok{\textgreater{}}
\end{Highlighting}
\end{Shaded}

\begin{center}\rule{0.5\linewidth}{0.5pt}\end{center}

\subsection{\texorpdfstring{Activate vs
\texttt{uv\ run}}{Activate vs uv run}}\label{activate-vs-uv-run}

\begin{itemize}
\tightlist
\item
  If you \textbf{activate}, \texttt{python} and \texttt{pip} point to
  the env
\item
  If you \textbf{don't activate}, \texttt{uv\ run\ ...} still uses the
  env
\end{itemize}

\textbf{Recommended habit:} use \texttt{uv\ run} for anything you want
to be reproducible

\marginnote{\begin{footnotesize}

You can activate the env, but \texttt{uv\ run} works even if you forget.

\end{footnotesize}}

\begin{center}\rule{0.5\linewidth}{0.5pt}\end{center}

\subsection{Create a new env (demo +
do)}\label{create-a-new-env-demo-do}

From inside \texttt{bootcamp/}:

\begin{Shaded}
\begin{Highlighting}[]
\ExtensionTok{uv}\NormalTok{ venv }\AttributeTok{{-}p}\NormalTok{ 3.11}
\end{Highlighting}
\end{Shaded}

Expected result: a folder named \texttt{.venv/}

\begin{center}\rule{0.5\linewidth}{0.5pt}\end{center}

\subsection{``Activate'' vs ``don't
activate''}\label{activate-vs-dont-activate}

\begin{itemize}
\tightlist
\item
  If you \textbf{activate}, \texttt{python} points to \texttt{.venv}
  automatically
\item
  If you \textbf{don't}, use \texttt{uv\ run\ ...} to guarantee the env
\end{itemize}

\begin{tcolorbox}[enhanced jigsaw, coltitle=black, bottomtitle=1mm, arc=.35mm, opacityback=0, colframe=quarto-callout-tip-color-frame, toptitle=1mm, colbacktitle=quarto-callout-tip-color!10!white, breakable, toprule=.15mm, titlerule=0mm, leftrule=.75mm, bottomrule=.15mm, rightrule=.15mm, title=\textcolor{quarto-callout-tip-color}{\faLightbulb}\hspace{0.5em}{Tip}, opacitybacktitle=0.6, left=2mm, colback=white]

If you ever wonder ``which python am I using?'', run
\texttt{which\ python} / \texttt{where\ python}.

\end{tcolorbox}

\begin{center}\rule{0.5\linewidth}{0.5pt}\end{center}

\subsection{Activate (optional but
useful)}\label{activate-optional-but-useful}

mac/linux:

\begin{Shaded}
\begin{Highlighting}[]
\BuiltInTok{.}\NormalTok{ .venv/bin/activate}
\end{Highlighting}
\end{Shaded}

Windows:

\begin{Shaded}
\begin{Highlighting}[]
\OperatorTok{.}\FunctionTok{venv}\NormalTok{\textbackslash{}Scripts\textbackslash{}activate}
\end{Highlighting}
\end{Shaded}

\textbf{Check:} your prompt usually changes.

\begin{center}\rule{0.5\linewidth}{0.5pt}\end{center}

\subsection{Install a package (we'll use
later)}\label{install-a-package-well-use-later}

\begin{Shaded}
\begin{Highlighting}[]
\ExtensionTok{uv}\NormalTok{ pip install typer streamlit}
\end{Highlighting}
\end{Shaded}

\begin{tcolorbox}[enhanced jigsaw, coltitle=black, bottomtitle=1mm, arc=.35mm, opacityback=0, colframe=quarto-callout-tip-color-frame, toptitle=1mm, colbacktitle=quarto-callout-tip-color!10!white, breakable, toprule=.15mm, titlerule=0mm, leftrule=.75mm, bottomrule=.15mm, rightrule=.15mm, title=\textcolor{quarto-callout-tip-color}{\faLightbulb}\hspace{0.5em}{Tip}, opacitybacktitle=0.6, left=2mm, colback=white]

If installation fails: copy the full error + your OS info and ask the
instructor.

\end{tcolorbox}

\begin{center}\rule{0.5\linewidth}{0.5pt}\end{center}

\subsection{\texorpdfstring{Run a Python script with
\texttt{uv\ run}}{Run a Python script with uv run}}\label{run-a-python-script-with-uv-run}

Create \texttt{hello.py}:

\begin{Shaded}
\begin{Highlighting}[]
\BuiltInTok{print}\NormalTok{(}\StringTok{"Hello from Week 1!"}\NormalTok{)}
\end{Highlighting}
\end{Shaded}

Run:

\begin{Shaded}
\begin{Highlighting}[]
\ExtensionTok{uv}\NormalTok{ run hello.py}
\end{Highlighting}
\end{Shaded}

\begin{center}\rule{0.5\linewidth}{0.5pt}\end{center}

\subsection{Quick Check}\label{quick-check}

What is the main difference?

\begin{itemize}
\tightlist
\item
  \begin{enumerate}
  \def\labelenumi{\Alph{enumi})}
  \tightlist
  \item
    \texttt{uv\ run\ hello.py}
  \end{enumerate}
\item
  \begin{enumerate}
  \def\labelenumi{\Alph{enumi})}
  \setcounter{enumi}{1}
  \tightlist
  \item
    \texttt{python\ hello.py}
  \end{enumerate}
\end{itemize}

. . .

\textbf{Answer:} \texttt{uv\ run} ensures the command runs inside the
project environment.

\begin{center}\rule{0.5\linewidth}{0.5pt}\end{center}

\subsection{Mini-lab: ``Run + break + fix'' (7
minutes)}\label{mini-lab-run-break-fix-7-minutes}

\begin{enumerate}
\def\labelenumi{\arabic{enumi}.}
\tightlist
\item
  Change \texttt{hello.py} to print your name
\item
  Introduce a syntax error (missing quote)
\item
  Run it and read the error
\item
  Fix it
\end{enumerate}

\textbf{Checkpoint:} you can explain what line the error points to.

\begin{center}\rule{0.5\linewidth}{0.5pt}\end{center}

\subsection{Session 1 recap}\label{session-1-recap}

\begin{itemize}
\tightlist
\item
  Terminal basics: \texttt{pwd}, \texttt{ls/dir}, \texttt{cd}
\item
  Paths and environment variables
\item
  \texttt{uv\ venv} + \texttt{uv\ run}
\end{itemize}

\section{Asr break}\label{asr-break}

\subsection{20 minutes}\label{minutes}

\textbf{When you return:} open VS Code in your \texttt{bootcamp/}
folder.

\section{Session 2}\label{session-2}

\subsection{Python values, containers,
operators}\label{python-values-containers-operators}

4:20pm--5:20pm

\begin{center}\rule{0.5\linewidth}{0.5pt}\end{center}

\subsection{Session 2 objectives}\label{session-2-objectives}

\begin{itemize}
\tightlist
\item
  Recognize Python's core value types
\item
  Use lists/tuples/sets/dicts
\item
  Use arithmetic, comparison, and logical operators
\item
  Predict the output of short expressions
\end{itemize}

\begin{center}\rule{0.5\linewidth}{0.5pt}\end{center}

\subsection{A Python program is just values +
steps}\label{a-python-program-is-just-values-steps}

\begin{enumerate}
\def\labelenumi{\arabic{enumi}.}
\tightlist
\item
  Create values (numbers, text, containers)
\item
  Combine them (operators)
\item
  Make decisions (if / loops)
\item
  Organize into functions and files
\end{enumerate}

\begin{center}\rule{0.5\linewidth}{0.5pt}\end{center}

\subsection{Literals: quick tour}\label{literals-quick-tour}

\begin{itemize}
\tightlist
\item
  \texttt{None}, \texttt{True}, \texttt{False}
\item
  Integers: \texttt{0}, \texttt{-2}, \texttt{1\_000\_000}, \texttt{0x1f}
\item
  Floats: \texttt{1.5}, \texttt{1e6}, \texttt{-2.5e-3}
\item
  Strings: \texttt{\textquotesingle{}hi\textquotesingle{}},
  \texttt{"hi"}, \texttt{"""multi"""}
\end{itemize}

\marginnote{\begin{footnotesize}

Underscores in numbers are allowed: \texttt{1\_000\_000}.

\end{footnotesize}}

\begin{center}\rule{0.5\linewidth}{0.5pt}\end{center}

\subsection{Containers (you'll use these all
week)}\label{containers-youll-use-these-all-week}

\begin{longtable}[]{@{}llll@{}}
\toprule\noalign{}
Type & Example & Mutable? & Typical use \\
\midrule\noalign{}
\endhead
\bottomrule\noalign{}
\endlastfoot
\texttt{list} & \texttt{{[}1,\ 2,\ 3{]}} & ✅ & ordered items \\
\texttt{tuple} & \texttt{(1,\ 2)} & ❌ & fixed group \\
\texttt{set} & \texttt{\{1,\ 2\}} & ✅ & unique items \\
\texttt{dict} & \texttt{\{\ "a":\ 1\ \}} & ✅ & key → value \\
\end{longtable}

\begin{center}\rule{0.5\linewidth}{0.5pt}\end{center}

\subsection{Tuples vs lists (when to use
which?)}\label{tuples-vs-lists-when-to-use-which}

\textbf{Tuple} (immutable)

\begin{Shaded}
\begin{Highlighting}[]
\NormalTok{point }\OperatorTok{=}\NormalTok{ (}\DecValTok{3}\NormalTok{, }\DecValTok{5}\NormalTok{)}
\end{Highlighting}
\end{Shaded}

\begin{itemize}
\tightlist
\item
  Fixed structure
\item
  Safe to pass around
\end{itemize}

\textbf{List} (mutable)

\begin{Shaded}
\begin{Highlighting}[]
\NormalTok{names }\OperatorTok{=}\NormalTok{ [}\StringTok{"Aisha"}\NormalTok{, }\StringTok{"Noor"}\NormalTok{]}
\NormalTok{names.append(}\StringTok{"Salem"}\NormalTok{)}
\end{Highlighting}
\end{Shaded}

\begin{itemize}
\tightlist
\item
  Grows/shrinks
\item
  Good for accumulation
\end{itemize}

\begin{center}\rule{0.5\linewidth}{0.5pt}\end{center}

\subsection{Tuples vs lists (quick
intuition)}\label{tuples-vs-lists-quick-intuition}

\begin{itemize}
\tightlist
\item
  Use a \textbf{list} when you plan to change it
\item
  Use a \textbf{tuple} for a fixed ``record'' (like coordinates)
\end{itemize}

\begin{Shaded}
\begin{Highlighting}[]
\NormalTok{point }\OperatorTok{=}\NormalTok{ (}\FloatTok{24.7136}\NormalTok{, }\FloatTok{46.6753}\NormalTok{)  }\CommentTok{\# (lat, lon)}
\NormalTok{names }\OperatorTok{=}\NormalTok{ [}\StringTok{"Aisha"}\NormalTok{, }\StringTok{"Fahad"}\NormalTok{]}
\NormalTok{names.append(}\StringTok{"Noor"}\NormalTok{)}
\end{Highlighting}
\end{Shaded}

\begin{center}\rule{0.5\linewidth}{0.5pt}\end{center}

\subsection{Sets: uniqueness tool}\label{sets-uniqueness-tool}

\begin{Shaded}
\begin{Highlighting}[]
\NormalTok{items }\OperatorTok{=}\NormalTok{ [}\StringTok{"a"}\NormalTok{, }\StringTok{"b"}\NormalTok{, }\StringTok{"b"}\NormalTok{, }\StringTok{"c"}\NormalTok{]}
\NormalTok{unique }\OperatorTok{=} \BuiltInTok{set}\NormalTok{(items)}
\BuiltInTok{print}\NormalTok{(unique)  }\CommentTok{\# \{\textquotesingle{}a\textquotesingle{},\textquotesingle{}b\textquotesingle{},\textquotesingle{}c\textquotesingle{}\} (order not guaranteed)}
\end{Highlighting}
\end{Shaded}

Use cases:

\begin{itemize}
\tightlist
\item
  remove duplicates
\item
  fast membership checks (\texttt{x\ in\ my\_set})
\end{itemize}

\begin{center}\rule{0.5\linewidth}{0.5pt}\end{center}

\subsection{Variables are labels, not
boxes}\label{variables-are-labels-not-boxes}

\begin{Shaded}
\begin{Highlighting}[]
\NormalTok{x }\OperatorTok{=}\NormalTok{ [}\DecValTok{1}\NormalTok{, }\DecValTok{2}\NormalTok{, }\DecValTok{3}\NormalTok{]}
\NormalTok{y }\OperatorTok{=}\NormalTok{ x}
\NormalTok{y.append(}\DecValTok{4}\NormalTok{)}
\BuiltInTok{print}\NormalTok{(x)  }\CommentTok{\# ?}
\end{Highlighting}
\end{Shaded}

. . .

\texttt{x} becomes \texttt{{[}1,\ 2,\ 3,\ 4{]}} because \texttt{x} and
\texttt{y} point to the same list.

\begin{center}\rule{0.5\linewidth}{0.5pt}\end{center}

\subsection{Operators: your everyday
toolkit}\label{operators-your-everyday-toolkit}

\begin{itemize}
\tightlist
\item
  Arithmetic: \texttt{+\ -\ *\ /\ //\ \%\ **}
\item
  Comparison:
  \texttt{\textless{}\ \textless{}=\ ==\ !=\ \textgreater{}=\ \textgreater{}}
\item
  Logical: \texttt{and\ or\ not}
\item
  Membership: \texttt{in}, \texttt{not\ in}
\item
  Identity: \texttt{is}, \texttt{is\ not} \emph{(usually with
  \texttt{None})}
\end{itemize}

\begin{center}\rule{0.5\linewidth}{0.5pt}\end{center}

\subsection{\texorpdfstring{\texttt{in} vs \texttt{==} vs
\texttt{is}}{in vs == vs is}}\label{in-vs-vs-is}

\begin{itemize}
\tightlist
\item
  \texttt{x\ in\ container} → membership (lists/strings/sets/dicts)
\item
  \texttt{x\ ==\ y} → value equality
\item
  \texttt{x\ is\ y} → same object in memory \emph{(use for
  \texttt{None})}
\end{itemize}

\begin{Shaded}
\begin{Highlighting}[]
\NormalTok{x }\OperatorTok{=} \VariableTok{None}
\ControlFlowTok{if}\NormalTok{ x }\KeywordTok{is} \VariableTok{None}\NormalTok{:}
    \BuiltInTok{print}\NormalTok{(}\StringTok{"missing"}\NormalTok{)}
\end{Highlighting}
\end{Shaded}

\begin{center}\rule{0.5\linewidth}{0.5pt}\end{center}

\subsection{Operator precedence (don't
guess)}\label{operator-precedence-dont-guess}

Rule of thumb:

\begin{enumerate}
\def\labelenumi{\arabic{enumi}.}
\tightlist
\item
  Parentheses \texttt{(...)}
\item
  Power \texttt{**}
\item
  Multiply/divide \texttt{*\ /\ //\ \%}
\item
  Add/subtract \texttt{+\ -}
\item
  Comparisons \texttt{==\ \textless{}\ \textgreater{}\ ...}
\item
  \texttt{not} → \texttt{and} → \texttt{or}
\end{enumerate}

When in doubt: add parentheses.

\begin{center}\rule{0.5\linewidth}{0.5pt}\end{center}

\subsection{Operator precedence (mental
model)}\label{operator-precedence-mental-model}

\begin{enumerate}
\def\labelenumi{\arabic{enumi}.}
\tightlist
\item
  Parentheses \texttt{(...)}
\item
  Exponents \texttt{**}
\item
  Multiply/divide \texttt{*\ /\ //\ \%}
\item
  Add/subtract \texttt{+\ -}
\item
  Comparisons \texttt{\textless{}\ ==\ \textgreater{}}
\item
  Logical \texttt{not}, \texttt{and}, \texttt{or}
\end{enumerate}

When in doubt: add parentheses.

\begin{center}\rule{0.5\linewidth}{0.5pt}\end{center}

\subsection{Quick Check: predict the
result}\label{quick-check-predict-the-result}

What do these evaluate to?

\begin{enumerate}
\def\labelenumi{\arabic{enumi}.}
\tightlist
\item
  \texttt{5\ //\ 2}
\item
  \texttt{5\ /\ 2}
\item
  \texttt{5\ \%\ 2}
\item
  \texttt{2\ **\ 3}
\end{enumerate}

. . .

Answers: \texttt{2}, \texttt{2.5}, \texttt{1}, \texttt{8}

\begin{center}\rule{0.5\linewidth}{0.5pt}\end{center}

\subsection{Truthiness (important for data
work)}\label{truthiness-important-for-data-work}

These are \textbf{False}:

\begin{itemize}
\tightlist
\item
  \texttt{None}, \texttt{0}, \texttt{0.0}, \texttt{""}, \texttt{{[}{]}},
  \texttt{\{\}}, \texttt{set()}
\end{itemize}

Most other things are \textbf{True}.

\begin{center}\rule{0.5\linewidth}{0.5pt}\end{center}

\subsection{Casting: turning text into
numbers}\label{casting-turning-text-into-numbers}

\begin{Shaded}
\begin{Highlighting}[]
\BuiltInTok{int}\NormalTok{(}\StringTok{"32"}\NormalTok{)}
\BuiltInTok{float}\NormalTok{(}\StringTok{"{-}2.5e{-}3"}\NormalTok{)}
\BuiltInTok{bool}\NormalTok{(}\StringTok{"False"}\NormalTok{)  }\CommentTok{\# careful!}
\end{Highlighting}
\end{Shaded}

\begin{tcolorbox}[enhanced jigsaw, coltitle=black, bottomtitle=1mm, arc=.35mm, opacityback=0, colframe=quarto-callout-warning-color-frame, toptitle=1mm, colbacktitle=quarto-callout-warning-color!10!white, breakable, toprule=.15mm, titlerule=0mm, leftrule=.75mm, bottomrule=.15mm, rightrule=.15mm, title=\textcolor{quarto-callout-warning-color}{\faExclamationTriangle}\hspace{0.5em}{Warning}, opacitybacktitle=0.6, left=2mm, colback=white]

\texttt{bool("False")} is \textbf{True} because it's a non-empty string.

\end{tcolorbox}

\begin{center}\rule{0.5\linewidth}{0.5pt}\end{center}

\subsection{Mini-quiz (pairs): casting}\label{mini-quiz-pairs-casting}

Decide \textbf{without running}:

\begin{enumerate}
\def\labelenumi{\arabic{enumi}.}
\tightlist
\item
  \texttt{bool({[}{]})}
\item
  \texttt{bool({[}0{]})}
\item
  \texttt{int(1.9)}
\item
  \texttt{float(3)}
\end{enumerate}

. . .

Answers: \texttt{False}, \texttt{True}, \texttt{1}, \texttt{3.0}

\begin{center}\rule{0.5\linewidth}{0.5pt}\end{center}

\subsection{Lists: indexing and
slicing}\label{lists-indexing-and-slicing}

\begin{Shaded}
\begin{Highlighting}[]
\NormalTok{x }\OperatorTok{=}\NormalTok{ [}\DecValTok{4}\NormalTok{, }\DecValTok{5}\NormalTok{, }\DecValTok{6}\NormalTok{, }\DecValTok{7}\NormalTok{, }\DecValTok{8}\NormalTok{, }\DecValTok{9}\NormalTok{]}
\BuiltInTok{print}\NormalTok{(x[}\DecValTok{1}\NormalTok{])     }\CommentTok{\# 5}
\BuiltInTok{print}\NormalTok{(x[}\OperatorTok{{-}}\DecValTok{2}\NormalTok{])    }\CommentTok{\# 8}
\BuiltInTok{print}\NormalTok{(x[}\DecValTok{1}\NormalTok{:}\DecValTok{3}\NormalTok{])   }\CommentTok{\# [5, 6]}
\BuiltInTok{print}\NormalTok{(x[::}\OperatorTok{{-}}\DecValTok{1}\NormalTok{])  }\CommentTok{\# reversed}
\end{Highlighting}
\end{Shaded}

\begin{center}\rule{0.5\linewidth}{0.5pt}\end{center}

\subsection{List comprehensions (loop →
list)}\label{list-comprehensions-loop-list}

\begin{Shaded}
\begin{Highlighting}[]
\NormalTok{nums }\OperatorTok{=}\NormalTok{ [}\DecValTok{1}\NormalTok{, }\DecValTok{2}\NormalTok{, }\DecValTok{3}\NormalTok{, }\DecValTok{4}\NormalTok{, }\DecValTok{5}\NormalTok{]}
\NormalTok{squares }\OperatorTok{=}\NormalTok{ [n }\OperatorTok{*}\NormalTok{ n }\ControlFlowTok{for}\NormalTok{ n }\KeywordTok{in}\NormalTok{ nums]}
\NormalTok{evens }\OperatorTok{=}\NormalTok{ [n }\ControlFlowTok{for}\NormalTok{ n }\KeywordTok{in}\NormalTok{ nums }\ControlFlowTok{if}\NormalTok{ n }\OperatorTok{\%} \DecValTok{2} \OperatorTok{==} \DecValTok{0}\NormalTok{]}
\end{Highlighting}
\end{Shaded}

Use when it's short and clear.

\begin{center}\rule{0.5\linewidth}{0.5pt}\end{center}

\subsection{Dicts: the ``data row'' type}\label{dicts-the-data-row-type}

\begin{Shaded}
\begin{Highlighting}[]
\NormalTok{row }\OperatorTok{=}\NormalTok{ \{}\StringTok{"name"}\NormalTok{: }\StringTok{"Aisha"}\NormalTok{, }\StringTok{"age"}\NormalTok{: }\DecValTok{23}\NormalTok{\}}
\BuiltInTok{print}\NormalTok{(row[}\StringTok{"name"}\NormalTok{])}
\NormalTok{row[}\StringTok{"age"}\NormalTok{] }\OperatorTok{+=} \DecValTok{1}
\end{Highlighting}
\end{Shaded}

Why we care:

\begin{itemize}
\tightlist
\item
  \texttt{csv.DictReader} gives you dicts (column → value)
\end{itemize}

\begin{center}\rule{0.5\linewidth}{0.5pt}\end{center}

\subsection{Micro-exercise: build a tiny ``row'' (6
minutes)}\label{micro-exercise-build-a-tiny-row-6-minutes}

Create a dictionary with:

\begin{itemize}
\tightlist
\item
  \texttt{"city"}
\item
  \texttt{"temp\_c"}
\item
  \texttt{"is\_weekend"}
\end{itemize}

Then print a sentence using an f-string.

\textbf{Checkpoint:} Your output includes all three values.

\begin{center}\rule{0.5\linewidth}{0.5pt}\end{center}

\subsection{Solution (example)}\label{solution-example}

\begin{Shaded}
\begin{Highlighting}[]
\NormalTok{row }\OperatorTok{=}\NormalTok{ \{}\StringTok{"city"}\NormalTok{: }\StringTok{"Riyadh"}\NormalTok{, }\StringTok{"temp\_c"}\NormalTok{: }\FloatTok{19.5}\NormalTok{, }\StringTok{"is\_weekend"}\NormalTok{: }\VariableTok{True}\NormalTok{\}}
\BuiltInTok{print}\NormalTok{(}\SpecialStringTok{f"In }\SpecialCharTok{\{}\NormalTok{row[}\StringTok{\textquotesingle{}city\textquotesingle{}}\NormalTok{]}\SpecialCharTok{\}}\SpecialStringTok{, temp is }\SpecialCharTok{\{}\NormalTok{row[}\StringTok{\textquotesingle{}temp\_c\textquotesingle{}}\NormalTok{]}\SpecialCharTok{\}}\SpecialStringTok{C. Weekend? }\SpecialCharTok{\{}\NormalTok{row[}\StringTok{\textquotesingle{}is\_weekend\textquotesingle{}}\NormalTok{]}\SpecialCharTok{\}}\SpecialStringTok{"}\NormalTok{)}
\end{Highlighting}
\end{Shaded}

\begin{center}\rule{0.5\linewidth}{0.5pt}\end{center}

\subsection{Session 2 recap}\label{session-2-recap}

\begin{itemize}
\tightlist
\item
  Values + containers
\item
  Operators and truthiness
\item
  Lists/dicts basics
\end{itemize}

\section{Maghrib break}\label{maghrib-break}

\subsection{20 minutes}\label{minutes-1}

\textbf{When you return:} we'll start writing real scripts and reading
files.

\section{Session 3}\label{session-3}

\subsection{Control flow + strings/lists/dicts +
files}\label{control-flow-stringslistsdicts-files}

5:40pm--6:40pm

\begin{center}\rule{0.5\linewidth}{0.5pt}\end{center}

\subsection{Session 3 objectives}\label{session-3-objectives}

\begin{itemize}
\tightlist
\item
  Use \texttt{if/elif/else} and conditional expressions
\item
  Loop with \texttt{for} and \texttt{while}
\item
  Handle errors with \texttt{try/except}
\item
  Read and write text files with \texttt{with\ open(...)}
\item
  Use strings + lists + dicts to build a report
\end{itemize}

\begin{center}\rule{0.5\linewidth}{0.5pt}\end{center}

\subsection{\texorpdfstring{\texttt{if} in
practice}{if in practice}}\label{if-in-practice}

\begin{Shaded}
\begin{Highlighting}[]
\NormalTok{grade }\OperatorTok{=} \DecValTok{83}
\ControlFlowTok{if}\NormalTok{ grade }\OperatorTok{\textgreater{}=} \DecValTok{90}\NormalTok{:}
\NormalTok{    letter }\OperatorTok{=} \StringTok{"A"}
\ControlFlowTok{elif}\NormalTok{ grade }\OperatorTok{\textgreater{}=} \DecValTok{80}\NormalTok{:}
\NormalTok{    letter }\OperatorTok{=} \StringTok{"B"}
\ControlFlowTok{else}\NormalTok{:}
\NormalTok{    letter }\OperatorTok{=} \StringTok{"C or below"}
\BuiltInTok{print}\NormalTok{(letter)}
\end{Highlighting}
\end{Shaded}

\begin{center}\rule{0.5\linewidth}{0.5pt}\end{center}

\subsection{One-line condition
(ternary)}\label{one-line-condition-ternary}

\begin{Shaded}
\begin{Highlighting}[]
\NormalTok{number }\OperatorTok{=} \DecValTok{7}
\NormalTok{parity }\OperatorTok{=} \StringTok{"even"} \ControlFlowTok{if}\NormalTok{ number }\OperatorTok{\%} \DecValTok{2} \OperatorTok{==} \DecValTok{0} \ControlFlowTok{else} \StringTok{"odd"}
\BuiltInTok{print}\NormalTok{(parity)}
\end{Highlighting}
\end{Shaded}

\begin{center}\rule{0.5\linewidth}{0.5pt}\end{center}

\subsection{\texorpdfstring{\texttt{for} loops: your default
loop}{for loops: your default loop}}\label{for-loops-your-default-loop}

\begin{Shaded}
\begin{Highlighting}[]
\NormalTok{names }\OperatorTok{=}\NormalTok{ [}\StringTok{"Aisha"}\NormalTok{, }\StringTok{"Fahad"}\NormalTok{, }\StringTok{"Noor"}\NormalTok{]}
\ControlFlowTok{for}\NormalTok{ name }\KeywordTok{in}\NormalTok{ names:}
    \BuiltInTok{print}\NormalTok{(name)}
\end{Highlighting}
\end{Shaded}

\textbf{Pattern you'll use for CSV:} loop over rows, update counters.

\begin{center}\rule{0.5\linewidth}{0.5pt}\end{center}

\subsection{\texorpdfstring{Loop helpers: \texttt{range},
\texttt{enumerate},
\texttt{zip}}{Loop helpers: range, enumerate, zip}}\label{loop-helpers-range-enumerate-zip}

\begin{Shaded}
\begin{Highlighting}[]
\ControlFlowTok{for}\NormalTok{ i }\KeywordTok{in} \BuiltInTok{range}\NormalTok{(}\DecValTok{3}\NormalTok{):}
    \BuiltInTok{print}\NormalTok{(i)}

\NormalTok{names }\OperatorTok{=}\NormalTok{ [}\StringTok{"Aisha"}\NormalTok{, }\StringTok{"Fahad"}\NormalTok{]}
\ControlFlowTok{for}\NormalTok{ i, name }\KeywordTok{in} \BuiltInTok{enumerate}\NormalTok{(names, start}\OperatorTok{=}\DecValTok{1}\NormalTok{):}
    \BuiltInTok{print}\NormalTok{(i, name)}

\NormalTok{ages }\OperatorTok{=}\NormalTok{ [}\DecValTok{23}\NormalTok{, }\DecValTok{31}\NormalTok{]}
\ControlFlowTok{for}\NormalTok{ name, age }\KeywordTok{in} \BuiltInTok{zip}\NormalTok{(names, ages):}
    \BuiltInTok{print}\NormalTok{(name, age)}
\end{Highlighting}
\end{Shaded}

\begin{center}\rule{0.5\linewidth}{0.5pt}\end{center}

\subsection{\texorpdfstring{\texttt{while} loops: when you don't know
``how
many''}{while loops: when you don't know ``how many''}}\label{while-loops-when-you-dont-know-how-many}

\begin{Shaded}
\begin{Highlighting}[]
\NormalTok{i }\OperatorTok{=} \DecValTok{0}
\ControlFlowTok{while}\NormalTok{ i }\OperatorTok{\textless{}} \DecValTok{3}\NormalTok{:}
    \BuiltInTok{print}\NormalTok{(i)}
\NormalTok{    i }\OperatorTok{+=} \DecValTok{1}
\end{Highlighting}
\end{Shaded}

\begin{center}\rule{0.5\linewidth}{0.5pt}\end{center}

\subsection{Common loop controls}\label{common-loop-controls}

\begin{itemize}
\tightlist
\item
  \texttt{continue} → skip to next iteration
\item
  \texttt{break} → stop the loop
\end{itemize}

\begin{Shaded}
\begin{Highlighting}[]
\ControlFlowTok{for}\NormalTok{ x }\KeywordTok{in} \StringTok{"abcdef"}\NormalTok{:}
    \ControlFlowTok{if}\NormalTok{ x }\OperatorTok{==} \StringTok{"b"}\NormalTok{:}
        \ControlFlowTok{continue}
    \ControlFlowTok{if}\NormalTok{ x }\OperatorTok{==} \StringTok{"e"}\NormalTok{:}
        \ControlFlowTok{break}
    \BuiltInTok{print}\NormalTok{(x)}
\end{Highlighting}
\end{Shaded}

\begin{center}\rule{0.5\linewidth}{0.5pt}\end{center}

\subsection{Quick Check}\label{quick-check-1}

What prints?

\begin{Shaded}
\begin{Highlighting}[]
\ControlFlowTok{for}\NormalTok{ x }\KeywordTok{in}\NormalTok{ [}\DecValTok{1}\NormalTok{, }\DecValTok{2}\NormalTok{, }\DecValTok{3}\NormalTok{, }\DecValTok{4}\NormalTok{]:}
    \ControlFlowTok{if}\NormalTok{ x }\OperatorTok{\%} \DecValTok{2} \OperatorTok{==} \DecValTok{0}\NormalTok{:}
        \ControlFlowTok{continue}
    \BuiltInTok{print}\NormalTok{(x)}
\end{Highlighting}
\end{Shaded}

. . .

\textbf{Answer:} \texttt{1} then \texttt{3}

\begin{center}\rule{0.5\linewidth}{0.5pt}\end{center}

\subsection{\texorpdfstring{\texttt{match}: clean branching (Python
3.10+)}{match: clean branching (Python 3.10+)}}\label{match-clean-branching-python-3.10}

\begin{Shaded}
\begin{Highlighting}[]
\NormalTok{cmd }\OperatorTok{=} \BuiltInTok{input}\NormalTok{(}\StringTok{"Command: "}\NormalTok{)}
\ControlFlowTok{match}\NormalTok{ cmd:}
    \ControlFlowTok{case} \StringTok{"stats"}\NormalTok{:}
        \BuiltInTok{print}\NormalTok{(}\StringTok{"Show stats"}\NormalTok{)}
    \ControlFlowTok{case} \StringTok{"help"}\NormalTok{:}
        \BuiltInTok{print}\NormalTok{(}\StringTok{"Show help"}\NormalTok{)}
    \ControlFlowTok{case}\NormalTok{ \_:}
        \BuiltInTok{print}\NormalTok{(}\StringTok{"Unknown command"}\NormalTok{)}
\end{Highlighting}
\end{Shaded}

\begin{center}\rule{0.5\linewidth}{0.5pt}\end{center}

\subsection{Assertions: enforce
assumptions}\label{assertions-enforce-assumptions}

\begin{Shaded}
\begin{Highlighting}[]
\NormalTok{age }\OperatorTok{=} \DecValTok{250}
\ControlFlowTok{assert} \DecValTok{0} \OperatorTok{\textless{}=}\NormalTok{ age }\OperatorTok{\textless{}=} \DecValTok{200}\NormalTok{, }\StringTok{"Age must be realistic"}
\end{Highlighting}
\end{Shaded}

Use this to catch ``impossible states'' early.

\begin{center}\rule{0.5\linewidth}{0.5pt}\end{center}

\subsection{Errors happen --- handle
them}\label{errors-happen-handle-them}

\begin{Shaded}
\begin{Highlighting}[]
\ControlFlowTok{try}\NormalTok{:}
\NormalTok{    x }\OperatorTok{=} \BuiltInTok{int}\NormalTok{(}\BuiltInTok{input}\NormalTok{(}\StringTok{"Enter a number: "}\NormalTok{))}
    \BuiltInTok{print}\NormalTok{(}\DecValTok{1} \OperatorTok{/}\NormalTok{ x)}
\ControlFlowTok{except} \PreprocessorTok{ValueError}\NormalTok{:}
    \BuiltInTok{print}\NormalTok{(}\StringTok{"That was not a number"}\NormalTok{)}
\ControlFlowTok{except} \PreprocessorTok{ZeroDivisionError}\NormalTok{:}
    \BuiltInTok{print}\NormalTok{(}\StringTok{"We cannot divide by zero"}\NormalTok{)}
\end{Highlighting}
\end{Shaded}

\begin{center}\rule{0.5\linewidth}{0.5pt}\end{center}

\subsection{\texorpdfstring{Files: always use
\texttt{with\ open(...)}}{Files: always use with open(...)}}\label{files-always-use-with-open...}

\begin{Shaded}
\begin{Highlighting}[]
\ControlFlowTok{with} \BuiltInTok{open}\NormalTok{(}\StringTok{"notes.txt"}\NormalTok{, mode}\OperatorTok{=}\StringTok{"w"}\NormalTok{) }\ImportTok{as}\NormalTok{ f:}
\NormalTok{    f.write(}\StringTok{"Hello file!}\CharTok{\textbackslash{}n}\StringTok{"}\NormalTok{)}

\ControlFlowTok{with} \BuiltInTok{open}\NormalTok{(}\StringTok{"notes.txt"}\NormalTok{, mode}\OperatorTok{=}\StringTok{"r"}\NormalTok{) }\ImportTok{as}\NormalTok{ f:}
    \BuiltInTok{print}\NormalTok{(f.read())}
\end{Highlighting}
\end{Shaded}

\begin{center}\rule{0.5\linewidth}{0.5pt}\end{center}

\subsection{Strings: indexing + methods}\label{strings-indexing-methods}

\begin{Shaded}
\begin{Highlighting}[]
\NormalTok{s }\OperatorTok{=} \StringTok{"  Data,Data,AI  "}
\BuiltInTok{print}\NormalTok{(s.strip())}
\BuiltInTok{print}\NormalTok{(s.lower())}
\BuiltInTok{print}\NormalTok{(s.split(}\StringTok{","}\NormalTok{))}
\end{Highlighting}
\end{Shaded}

\begin{center}\rule{0.5\linewidth}{0.5pt}\end{center}

\subsection{f-strings: readable
formatting}\label{f-strings-readable-formatting}

\begin{Shaded}
\begin{Highlighting}[]
\NormalTok{name }\OperatorTok{=} \StringTok{"Noor"}
\NormalTok{score }\OperatorTok{=} \FloatTok{91.23456}
\BuiltInTok{print}\NormalTok{(}\SpecialStringTok{f"}\SpecialCharTok{\{}\NormalTok{name}\SpecialCharTok{\}}\SpecialStringTok{ scored }\SpecialCharTok{\{}\NormalTok{score}\SpecialCharTok{:.2f\}}\SpecialStringTok{"}\NormalTok{)}

\end{Highlighting}
\end{Shaded}

\begin{center}\rule{0.5\linewidth}{0.5pt}\end{center}

\subsection{Built-ins you'll use in
profilers}\label{built-ins-youll-use-in-profilers}

\begin{itemize}
\tightlist
\item
  \texttt{len(rows)} → row count
\item
  \texttt{min(numbers)}, \texttt{max(numbers)} → extremes
\item
  \texttt{sum(numbers)\ /\ len(numbers)} → mean
\item
  \texttt{sorted(items)} → ordering
\item
  \texttt{enumerate(items)} → index + value
\end{itemize}

\begin{Shaded}
\begin{Highlighting}[]
\NormalTok{names }\OperatorTok{=}\NormalTok{ [}\StringTok{"Aisha"}\NormalTok{, }\StringTok{"Fahad"}\NormalTok{]}
\ControlFlowTok{for}\NormalTok{ i, name }\KeywordTok{in} \BuiltInTok{enumerate}\NormalTok{(names, start}\OperatorTok{=}\DecValTok{1}\NormalTok{):}
    \BuiltInTok{print}\NormalTok{(i, name)}
\end{Highlighting}
\end{Shaded}

\marginnote{\begin{footnotesize}

We'll add numeric stats tomorrow.

\end{footnotesize}}

\begin{center}\rule{0.5\linewidth}{0.5pt}\end{center}

\subsection{Lists + dicts: the report builder
pattern}\label{lists-dicts-the-report-builder-pattern}

You will build a report like:

\begin{Shaded}
\begin{Highlighting}[]
\NormalTok{report }\OperatorTok{=}\NormalTok{ \{}
  \StringTok{"rows"}\NormalTok{: }\DecValTok{120}\NormalTok{,}
  \StringTok{"columns"}\NormalTok{: \{}
     \StringTok{"age"}\NormalTok{: \{}\StringTok{"missing"}\NormalTok{: }\DecValTok{2}\NormalTok{, }\StringTok{"type"}\NormalTok{: }\StringTok{"number"}\NormalTok{\},}
     \StringTok{"city"}\NormalTok{: \{}\StringTok{"missing"}\NormalTok{: }\DecValTok{0}\NormalTok{, }\StringTok{"type"}\NormalTok{: }\StringTok{"text"}\NormalTok{\}}
\NormalTok{  \}}
\NormalTok{\}}
\end{Highlighting}
\end{Shaded}

\begin{center}\rule{0.5\linewidth}{0.5pt}\end{center}

\subsection{Micro-exercise: count missing values (8
minutes)}\label{micro-exercise-count-missing-values-8-minutes}

Given this list of rows:

\begin{Shaded}
\begin{Highlighting}[]
\NormalTok{rows }\OperatorTok{=}\NormalTok{ [}
\NormalTok{  \{}\StringTok{"age"}\NormalTok{: }\StringTok{"19"}\NormalTok{, }\StringTok{"city"}\NormalTok{: }\StringTok{"Riyadh"}\NormalTok{\},}
\NormalTok{  \{}\StringTok{"age"}\NormalTok{: }\StringTok{""}\NormalTok{,   }\StringTok{"city"}\NormalTok{: }\StringTok{"Jeddah"}\NormalTok{\},}
\NormalTok{  \{}\StringTok{"age"}\NormalTok{: }\StringTok{"20"}\NormalTok{, }\StringTok{"city"}\NormalTok{: }\StringTok{""}\NormalTok{\},}
\NormalTok{]}
\end{Highlighting}
\end{Shaded}

Write code that counts missing values \textbf{per column}.

\textbf{Rule:} treat empty string \texttt{""} as missing.

\begin{center}\rule{0.5\linewidth}{0.5pt}\end{center}

\subsection{Solution (one good
approach)}\label{solution-one-good-approach}

\begin{Shaded}
\begin{Highlighting}[]
\NormalTok{missing }\OperatorTok{=}\NormalTok{ \{}\StringTok{"age"}\NormalTok{: }\DecValTok{0}\NormalTok{, }\StringTok{"city"}\NormalTok{: }\DecValTok{0}\NormalTok{\}}
\ControlFlowTok{for}\NormalTok{ row }\KeywordTok{in}\NormalTok{ rows:}
    \ControlFlowTok{for}\NormalTok{ col }\KeywordTok{in}\NormalTok{ missing:}
        \ControlFlowTok{if}\NormalTok{ row[col] }\OperatorTok{==} \StringTok{""}\NormalTok{:}
\NormalTok{            missing[col] }\OperatorTok{+=} \DecValTok{1}
\BuiltInTok{print}\NormalTok{(missing)}
\end{Highlighting}
\end{Shaded}

\begin{center}\rule{0.5\linewidth}{0.5pt}\end{center}

\subsection{Session 3 recap}\label{session-3-recap}

\begin{itemize}
\tightlist
\item
  Control flow: if / loops / match
\item
  Files: \texttt{with\ open(...)}
\item
  Report pattern: nested dicts
\end{itemize}

\section{Isha break}\label{isha-break}

\subsection{20 minutes}\label{minutes-2}

\textbf{When you return:} we start building the weekly project.

\section{Hands-on}\label{hands-on}

\subsection{Build the project: CSV Profiler (Part
1)}\label{build-the-project-csv-profiler-part-1}

7:00pm--9:00pm

\begin{center}\rule{0.5\linewidth}{0.5pt}\end{center}

\subsection{Hands-on success criteria
(today)}\label{hands-on-success-criteria-today}

By 9:00pm you should have:

\begin{itemize}
\tightlist
\item
  A project folder with \texttt{.venv/}
\item
  A Python package \texttt{csv\_profiler/}
\item
  Code that:

  \begin{itemize}
  \tightlist
  \item
    reads a CSV
  \item
    computes basic profile (rows/cols, missing counts)
  \item
    writes \texttt{report.json} and \texttt{report.md}
  \end{itemize}
\end{itemize}

\begin{center}\rule{0.5\linewidth}{0.5pt}\end{center}

\subsection{Weekly project folder layout
(target)}\label{weekly-project-folder-layout-target}

\begin{Shaded}
\begin{Highlighting}[]
\NormalTok{bootcamp/}
\NormalTok{  csv{-}profiler/}
\NormalTok{    pyproject.toml}
\NormalTok{    README.md}
\NormalTok{    src/}
\NormalTok{      csv\_profiler/}
\NormalTok{        \_\_init\_\_.py}
\NormalTok{        io.py}
\NormalTok{        profile.py}
\NormalTok{        render.py}
\NormalTok{        cli.py}
\NormalTok{    data/}
\NormalTok{      sample.csv}
\NormalTok{    outputs/}
\NormalTok{      report.json}
\NormalTok{      report.md}
\end{Highlighting}
\end{Shaded}

\marginnote{\begin{footnotesize}

Today we'll create a minimal version. We'll improve structure during the
week.

\end{footnotesize}}

\begin{center}\rule{0.5\linewidth}{0.5pt}\end{center}

\subsection{Task 0 --- Create the project (10
minutes)}\label{task-0-create-the-project-10-minutes}

\begin{enumerate}
\def\labelenumi{\arabic{enumi}.}
\tightlist
\item
  Inside \texttt{bootcamp/}, create a folder \texttt{csv-profiler/}
\item
  \texttt{cd\ csv-profiler}
\item
  Create an env: \texttt{uv\ venv\ -p\ 3.11}
\end{enumerate}

\textbf{Checkpoint:} you have \texttt{.venv/} inside
\texttt{csv-profiler/}.

\begin{center}\rule{0.5\linewidth}{0.5pt}\end{center}

\subsection{Task 0 --- Suggested
commands}\label{task-0-suggested-commands}

\begin{Shaded}
\begin{Highlighting}[]
\BuiltInTok{cd}\NormalTok{ \textasciitilde{}/bootcamp}
\FunctionTok{mkdir}\NormalTok{ csv{-}profiler}
\BuiltInTok{cd}\NormalTok{ csv{-}profiler}
\ExtensionTok{uv}\NormalTok{ venv }\AttributeTok{{-}p}\NormalTok{ 3.11}
\end{Highlighting}
\end{Shaded}

\begin{center}\rule{0.5\linewidth}{0.5pt}\end{center}

\subsection{Task 1 --- Add a sample CSV (8
minutes)}\label{task-1-add-a-sample-csv-8-minutes}

Create \texttt{data/sample.csv} with this content:

\begin{Shaded}
\begin{Highlighting}[]
\NormalTok{name,age,city,salary}
\NormalTok{Aisha,23,Riyadh,12000}
\NormalTok{Fahad,,Jeddah,9000}
\NormalTok{Noor,29,,}
\NormalTok{Salem,31,Dammam,15000}
\end{Highlighting}
\end{Shaded}

\textbf{Checkpoint:} you can open it in VS Code.

\begin{center}\rule{0.5\linewidth}{0.5pt}\end{center}

\subsection{Task 2 --- Create package skeleton (10
minutes)}\label{task-2-create-package-skeleton-10-minutes}

Create these files:

\begin{itemize}
\tightlist
\item
  \texttt{src/csv\_profiler/\_\_init\_\_.py}
\item
  \texttt{src/csv\_profiler/io.py}
\item
  \texttt{src/csv\_profiler/profile.py}
\item
  \texttt{src/csv\_profiler/render.py}
\item
  \texttt{main.py} \emph{(temporary entrypoint for today)}
\end{itemize}

\begin{center}\rule{0.5\linewidth}{0.5pt}\end{center}

\subsection{\texorpdfstring{Task 2 --- Minimal
\texttt{main.py}}{Task 2 --- Minimal main.py}}\label{task-2-minimal-main.py}

Paste this first:

\begin{Shaded}
\begin{Highlighting}[]
\ImportTok{from}\NormalTok{ csv\_profiler.io }\ImportTok{import}\NormalTok{ read\_csv\_rows}
\ImportTok{from}\NormalTok{ csv\_profiler.profile }\ImportTok{import}\NormalTok{ basic\_profile}
\ImportTok{from}\NormalTok{ csv\_profiler.render }\ImportTok{import}\NormalTok{ write\_json, write\_markdown}


\KeywordTok{def}\NormalTok{ main() }\OperatorTok{{-}\textgreater{}} \VariableTok{None}\NormalTok{:}
\NormalTok{    rows }\OperatorTok{=}\NormalTok{ read\_csv\_rows(}\StringTok{"data/sample.csv"}\NormalTok{)}
\NormalTok{    report }\OperatorTok{=}\NormalTok{ basic\_profile(rows)}
\NormalTok{    write\_json(report, }\StringTok{"outputs/report.json"}\NormalTok{)}
\NormalTok{    write\_markdown(report, }\StringTok{"outputs/report.md"}\NormalTok{)}
    \BuiltInTok{print}\NormalTok{(}\StringTok{"Wrote outputs/report.json and outputs/report.md"}\NormalTok{)}


\ControlFlowTok{if} \VariableTok{\_\_name\_\_} \OperatorTok{==} \StringTok{"\_\_main\_\_"}\NormalTok{:}
\NormalTok{    main()}
\end{Highlighting}
\end{Shaded}

\begin{center}\rule{0.5\linewidth}{0.5pt}\end{center}

\subsection{Task 3 --- Implement CSV reading (15
minutes)}\label{task-3-implement-csv-reading-15-minutes}

In \texttt{src/csv\_profiler/io.py} implement:

\begin{Shaded}
\begin{Highlighting}[]
\ImportTok{from}\NormalTok{ \_\_future\_\_ }\ImportTok{import}\NormalTok{ annotations}

\ImportTok{from}\NormalTok{ csv }\ImportTok{import}\NormalTok{ DictReader}
\ImportTok{from}\NormalTok{ pathlib }\ImportTok{import}\NormalTok{ Path}


\KeywordTok{def}\NormalTok{ read\_csv\_rows(path: }\BuiltInTok{str} \OperatorTok{|}\NormalTok{ Path) }\OperatorTok{{-}\textgreater{}} \BuiltInTok{list}\NormalTok{[}\BuiltInTok{dict}\NormalTok{[}\BuiltInTok{str}\NormalTok{, }\BuiltInTok{str}\NormalTok{]]:}
    \CommentTok{"""Read a CSV as a list of rows (each row is a dict of strings)."""}
\NormalTok{    ...}
\end{Highlighting}
\end{Shaded}

Rules:

\begin{itemize}
\tightlist
\item
  Use \texttt{with\ open(...,\ newline="")}
\item
  Return \texttt{list{[}dict{[}str,\ str{]}{]}}
\end{itemize}

\begin{center}\rule{0.5\linewidth}{0.5pt}\end{center}

\subsection{\texorpdfstring{Solution --- \texttt{read\_csv\_rows}
(example)}{Solution --- read\_csv\_rows (example)}}\label{solution-read_csv_rows-example}

\begin{Shaded}
\begin{Highlighting}[]
\ImportTok{from}\NormalTok{ \_\_future\_\_ }\ImportTok{import}\NormalTok{ annotations}

\ImportTok{from}\NormalTok{ csv }\ImportTok{import}\NormalTok{ DictReader}
\ImportTok{from}\NormalTok{ pathlib }\ImportTok{import}\NormalTok{ Path}


\KeywordTok{def}\NormalTok{ read\_csv\_rows(path: }\BuiltInTok{str} \OperatorTok{|}\NormalTok{ Path) }\OperatorTok{{-}\textgreater{}} \BuiltInTok{list}\NormalTok{[}\BuiltInTok{dict}\NormalTok{[}\BuiltInTok{str}\NormalTok{, }\BuiltInTok{str}\NormalTok{]]:}
\NormalTok{    path }\OperatorTok{=}\NormalTok{ Path(path)}
    \ControlFlowTok{with}\NormalTok{ path.}\BuiltInTok{open}\NormalTok{(}\StringTok{"r"}\NormalTok{, encoding}\OperatorTok{=}\StringTok{"utf{-}8"}\NormalTok{, newline}\OperatorTok{=}\StringTok{""}\NormalTok{) }\ImportTok{as}\NormalTok{ f:}
\NormalTok{        reader }\OperatorTok{=}\NormalTok{ DictReader(f)}
        \ControlFlowTok{return}\NormalTok{ [}\BuiltInTok{dict}\NormalTok{(row) }\ControlFlowTok{for}\NormalTok{ row }\KeywordTok{in}\NormalTok{ reader]}
\end{Highlighting}
\end{Shaded}

\begin{center}\rule{0.5\linewidth}{0.5pt}\end{center}

\subsection{Task 4 --- Compute a basic profile (20
minutes)}\label{task-4-compute-a-basic-profile-20-minutes}

In \texttt{src/csv\_profiler/profile.py}, implement:

\begin{Shaded}
\begin{Highlighting}[]
\KeywordTok{def}\NormalTok{ basic\_profile(rows: }\BuiltInTok{list}\NormalTok{[}\BuiltInTok{dict}\NormalTok{[}\BuiltInTok{str}\NormalTok{, }\BuiltInTok{str}\NormalTok{]]) }\OperatorTok{{-}\textgreater{}} \BuiltInTok{dict}\NormalTok{:}
    \CommentTok{"""Compute row count, column names, and missing values per column."""}
\NormalTok{    ...}
\end{Highlighting}
\end{Shaded}

Definition of \textbf{missing} today:

\begin{itemize}
\tightlist
\item
  empty string after stripping whitespace
\end{itemize}

\begin{center}\rule{0.5\linewidth}{0.5pt}\end{center}

\subsection{Hint --- how to get columns}\label{hint-how-to-get-columns}

If there is at least one row:

\begin{Shaded}
\begin{Highlighting}[]
\NormalTok{columns }\OperatorTok{=} \BuiltInTok{list}\NormalTok{(rows[}\DecValTok{0}\NormalTok{].keys())}
\end{Highlighting}
\end{Shaded}

Then loop rows and update counts.

\begin{center}\rule{0.5\linewidth}{0.5pt}\end{center}

\subsection{\texorpdfstring{Solution --- \texttt{basic\_profile} (day-1
version)}{Solution --- basic\_profile (day-1 version)}}\label{solution-basic_profile-day-1-version}

\begin{Shaded}
\begin{Highlighting}[]
\KeywordTok{def}\NormalTok{ basic\_profile(rows: }\BuiltInTok{list}\NormalTok{[}\BuiltInTok{dict}\NormalTok{[}\BuiltInTok{str}\NormalTok{, }\BuiltInTok{str}\NormalTok{]]) }\OperatorTok{{-}\textgreater{}} \BuiltInTok{dict}\NormalTok{:}
    \ControlFlowTok{if} \KeywordTok{not}\NormalTok{ rows:}
        \ControlFlowTok{return}\NormalTok{ \{}\StringTok{"rows"}\NormalTok{: }\DecValTok{0}\NormalTok{, }\StringTok{"columns"}\NormalTok{: \{\}, }\StringTok{"notes"}\NormalTok{: [}\StringTok{"Empty dataset"}\NormalTok{]\}}

\NormalTok{    columns }\OperatorTok{=} \BuiltInTok{list}\NormalTok{(rows[}\DecValTok{0}\NormalTok{].keys())}
\NormalTok{    missing }\OperatorTok{=}\NormalTok{ \{c: }\DecValTok{0} \ControlFlowTok{for}\NormalTok{ c }\KeywordTok{in}\NormalTok{ columns\}}
\NormalTok{    non\_empty }\OperatorTok{=}\NormalTok{ \{c: }\DecValTok{0} \ControlFlowTok{for}\NormalTok{ c }\KeywordTok{in}\NormalTok{ columns\}}

    \ControlFlowTok{for}\NormalTok{ row }\KeywordTok{in}\NormalTok{ rows:}
        \ControlFlowTok{for}\NormalTok{ c }\KeywordTok{in}\NormalTok{ columns:}
\NormalTok{            v }\OperatorTok{=}\NormalTok{ (row.get(c) }\KeywordTok{or} \StringTok{""}\NormalTok{).strip()}
            \ControlFlowTok{if}\NormalTok{ v }\OperatorTok{==} \StringTok{""}\NormalTok{:}
\NormalTok{                missing[c] }\OperatorTok{+=} \DecValTok{1}
            \ControlFlowTok{else}\NormalTok{:}
\NormalTok{                non\_empty[c] }\OperatorTok{+=} \DecValTok{1}

    \ControlFlowTok{return}\NormalTok{ \{}
        \StringTok{"rows"}\NormalTok{: }\BuiltInTok{len}\NormalTok{(rows),}
        \StringTok{"n\_cols"}\NormalTok{: }\BuiltInTok{len}\NormalTok{(columns),}
        \StringTok{"columns"}\NormalTok{: columns,}
        \StringTok{"missing"}\NormalTok{: missing,}
        \StringTok{"non\_empty"}\NormalTok{: non\_empty,}
\NormalTok{    \}}
\end{Highlighting}
\end{Shaded}

\begin{center}\rule{0.5\linewidth}{0.5pt}\end{center}

\subsection{Optional (stretch): start type
inference}\label{optional-stretch-start-type-inference}

Goal: infer a simple type label per column:

\begin{itemize}
\tightlist
\item
  \texttt{number} if all non-empty values can be parsed as float
\item
  \texttt{text} otherwise
\end{itemize}

Pseudo-steps:

\begin{enumerate}
\def\labelenumi{\arabic{enumi}.}
\tightlist
\item
  For each column, collect its non-empty strings
\item
  Try \texttt{float(value)} in a \texttt{try/except\ ValueError}
\item
  If any value fails → \texttt{text}
\end{enumerate}

Example helper:

\begin{Shaded}
\begin{Highlighting}[]
\KeywordTok{def}\NormalTok{ is\_number(s: }\BuiltInTok{str}\NormalTok{) }\OperatorTok{{-}\textgreater{}} \BuiltInTok{bool}\NormalTok{:}
    \ControlFlowTok{try}\NormalTok{:}
        \BuiltInTok{float}\NormalTok{(s)}
        \ControlFlowTok{return} \VariableTok{True}
    \ControlFlowTok{except} \PreprocessorTok{ValueError}\NormalTok{:}
        \ControlFlowTok{return} \VariableTok{False}
\end{Highlighting}
\end{Shaded}

\begin{center}\rule{0.5\linewidth}{0.5pt}\end{center}

\subsection{Task 5 --- Write JSON output (10
minutes)}\label{task-5-write-json-output-10-minutes}

In \texttt{src/csv\_profiler/render.py} implement:

\begin{Shaded}
\begin{Highlighting}[]
\ImportTok{from}\NormalTok{ \_\_future\_\_ }\ImportTok{import}\NormalTok{ annotations}

\ImportTok{from}\NormalTok{ pathlib }\ImportTok{import}\NormalTok{ Path}


\KeywordTok{def}\NormalTok{ write\_json(report: }\BuiltInTok{dict}\NormalTok{, path: }\BuiltInTok{str} \OperatorTok{|}\NormalTok{ Path) }\OperatorTok{{-}\textgreater{}} \VariableTok{None}\NormalTok{:}
\NormalTok{    ...}
\end{Highlighting}
\end{Shaded}

Requirements:

\begin{itemize}
\tightlist
\item
  Create the parent folder if it doesn't exist
\item
  Pretty-print with indentation
\end{itemize}

\begin{center}\rule{0.5\linewidth}{0.5pt}\end{center}

\subsection{\texorpdfstring{Solution ---
\texttt{write\_json}}{Solution --- write\_json}}\label{solution-write_json}

\begin{Shaded}
\begin{Highlighting}[]
\ImportTok{from}\NormalTok{ \_\_future\_\_ }\ImportTok{import}\NormalTok{ annotations}

\ImportTok{import}\NormalTok{ json}
\ImportTok{from}\NormalTok{ pathlib }\ImportTok{import}\NormalTok{ Path}


\KeywordTok{def}\NormalTok{ write\_json(report: }\BuiltInTok{dict}\NormalTok{, path: }\BuiltInTok{str} \OperatorTok{|}\NormalTok{ Path) }\OperatorTok{{-}\textgreater{}} \VariableTok{None}\NormalTok{:}
\NormalTok{    path }\OperatorTok{=}\NormalTok{ Path(path)}
\NormalTok{    path.parent.mkdir(parents}\OperatorTok{=}\VariableTok{True}\NormalTok{, exist\_ok}\OperatorTok{=}\VariableTok{True}\NormalTok{)}
\NormalTok{    path.write\_text(json.dumps(report, indent}\OperatorTok{=}\DecValTok{2}\NormalTok{, ensure\_ascii}\OperatorTok{=}\VariableTok{False}\NormalTok{) }\OperatorTok{+} \StringTok{"}\CharTok{\textbackslash{}n}\StringTok{"}\NormalTok{, encoding}\OperatorTok{=}\StringTok{"utf{-}8"}\NormalTok{)}
\end{Highlighting}
\end{Shaded}

\begin{center}\rule{0.5\linewidth}{0.5pt}\end{center}

\subsection{Task 6 --- Write Markdown output (15
minutes)}\label{task-6-write-markdown-output-15-minutes}

Implement:

\begin{Shaded}
\begin{Highlighting}[]
\KeywordTok{def}\NormalTok{ write\_markdown(report: }\BuiltInTok{dict}\NormalTok{, path: }\BuiltInTok{str} \OperatorTok{|}\NormalTok{ Path) }\OperatorTok{{-}\textgreater{}} \VariableTok{None}\NormalTok{:}
\NormalTok{    ...}
\end{Highlighting}
\end{Shaded}

Markdown should include:

\begin{itemize}
\tightlist
\item
  Title
\item
  Rows + columns
\item
  A small table: column name + missing count
\end{itemize}

\begin{center}\rule{0.5\linewidth}{0.5pt}\end{center}

\subsection{\texorpdfstring{Solution --- \texttt{write\_markdown}
(simple)}{Solution --- write\_markdown (simple)}}\label{solution-write_markdown-simple}

\begin{Shaded}
\begin{Highlighting}[]
\ImportTok{from}\NormalTok{ \_\_future\_\_ }\ImportTok{import}\NormalTok{ annotations}

\ImportTok{from}\NormalTok{ pathlib }\ImportTok{import}\NormalTok{ Path}


\KeywordTok{def}\NormalTok{ write\_markdown(report: }\BuiltInTok{dict}\NormalTok{, path: }\BuiltInTok{str} \OperatorTok{|}\NormalTok{ Path) }\OperatorTok{{-}\textgreater{}} \VariableTok{None}\NormalTok{:}
\NormalTok{    path }\OperatorTok{=}\NormalTok{ Path(path)}
\NormalTok{    path.parent.mkdir(parents}\OperatorTok{=}\VariableTok{True}\NormalTok{, exist\_ok}\OperatorTok{=}\VariableTok{True}\NormalTok{)}

\NormalTok{    cols }\OperatorTok{=}\NormalTok{ report.get(}\StringTok{"columns"}\NormalTok{, [])}
\NormalTok{    missing }\OperatorTok{=}\NormalTok{ report.get(}\StringTok{"missing"}\NormalTok{, \{\})}
\NormalTok{    lines: }\BuiltInTok{list}\NormalTok{[}\BuiltInTok{str}\NormalTok{] }\OperatorTok{=}\NormalTok{ []}
\NormalTok{    lines.append(}\StringTok{"\# CSV Profiling Report}\CharTok{\textbackslash{}n}\StringTok{"}\NormalTok{)}
\NormalTok{    lines.append(}\SpecialStringTok{f"{-} Rows: **}\SpecialCharTok{\{}\NormalTok{report}\SpecialCharTok{.}\NormalTok{get(}\StringTok{\textquotesingle{}rows\textquotesingle{}}\NormalTok{, }\DecValTok{0}\NormalTok{)}\SpecialCharTok{\}}\SpecialStringTok{**"}\NormalTok{)}
\NormalTok{    lines.append(}\SpecialStringTok{f"{-} Columns: **}\SpecialCharTok{\{}\NormalTok{report}\SpecialCharTok{.}\NormalTok{get(}\StringTok{\textquotesingle{}n\_cols\textquotesingle{}}\NormalTok{, }\DecValTok{0}\NormalTok{)}\SpecialCharTok{\}}\SpecialStringTok{**}\CharTok{\textbackslash{}n}\SpecialStringTok{"}\NormalTok{)}

\NormalTok{    lines.append(}\StringTok{"\#\# Missing Values}\CharTok{\textbackslash{}n}\StringTok{"}\NormalTok{)}
\NormalTok{    lines.append(}\StringTok{"| column | missing |"}\NormalTok{)}
\NormalTok{    lines.append(}\StringTok{"|{-}{-}{-}|{-}{-}{-}:|"}\NormalTok{)}
    \ControlFlowTok{for}\NormalTok{ c }\KeywordTok{in}\NormalTok{ cols:}
\NormalTok{        lines.append(}\SpecialStringTok{f"| }\SpecialCharTok{\{}\NormalTok{c}\SpecialCharTok{\}}\SpecialStringTok{ | }\SpecialCharTok{\{}\NormalTok{missing}\SpecialCharTok{.}\NormalTok{get(c, }\DecValTok{0}\NormalTok{)}\SpecialCharTok{\}}\SpecialStringTok{ |"}\NormalTok{)}

\NormalTok{    path.write\_text(}\StringTok{"}\CharTok{\textbackslash{}n}\StringTok{"}\NormalTok{.join(lines) }\OperatorTok{+} \StringTok{"}\CharTok{\textbackslash{}n}\StringTok{"}\NormalTok{, encoding}\OperatorTok{=}\StringTok{"utf{-}8"}\NormalTok{)}
\end{Highlighting}
\end{Shaded}

\begin{center}\rule{0.5\linewidth}{0.5pt}\end{center}

\subsection{Task 7 --- Run it end-to-end (10
minutes)}\label{task-7-run-it-end-to-end-10-minutes}

From the project root:

\begin{Shaded}
\begin{Highlighting}[]
\ExtensionTok{uv}\NormalTok{ run main.py}
\end{Highlighting}
\end{Shaded}

Then open:

\begin{itemize}
\tightlist
\item
  \texttt{outputs/report.json}
\item
  \texttt{outputs/report.md}
\end{itemize}

\textbf{Checkpoint:} both files exist and match the sample CSV.

\begin{center}\rule{0.5\linewidth}{0.5pt}\end{center}

\subsection{Debug playbook (when it
fails)}\label{debug-playbook-when-it-fails}

\begin{enumerate}
\def\labelenumi{\arabic{enumi}.}
\tightlist
\item
  Read the \textbf{first error line} (most important)
\item
  Confirm the file path is correct
\item
  Print intermediate values (\texttt{print(rows{[}0{]})})
\item
  Reduce the problem (try 1 row)
\end{enumerate}

\begin{center}\rule{0.5\linewidth}{0.5pt}\end{center}

\subsection{Stretch (if you finish
early)}\label{stretch-if-you-finish-early}

Add one more section to the Markdown:

\begin{itemize}
\tightlist
\item
  \textbf{Non-empty counts} per column
\end{itemize}

Bonus:

\begin{itemize}
\tightlist
\item
  A \texttt{top\_values} list for text columns (most common values)
\end{itemize}

\begin{center}\rule{0.5\linewidth}{0.5pt}\end{center}

\subsection{Exit Ticket}\label{exit-ticket}

In 1--2 sentences:

\textbf{What was the most confusing part today: paths, environments, or
Python control flow?}

\begin{center}\rule{0.5\linewidth}{0.5pt}\end{center}

\subsection{What to do after class (Day 1
assignment)}\label{what-to-do-after-class-day-1-assignment}

\textbf{Due:} before Day 2 starts

\begin{enumerate}
\def\labelenumi{\arabic{enumi}.}
\tightlist
\item
  Make your code work on \texttt{data/sample.csv}
\item
  Change the sample CSV (add 2 rows) and rerun
\item
  Update \texttt{report.md} to include a short ``Notes'' section
\end{enumerate}

\textbf{Deliverable:} a zip or folder with your \texttt{csv-profiler/}
project.

\begin{tcolorbox}[enhanced jigsaw, coltitle=black, bottomtitle=1mm, arc=.35mm, opacityback=0, colframe=quarto-callout-tip-color-frame, toptitle=1mm, colbacktitle=quarto-callout-tip-color!10!white, breakable, toprule=.15mm, titlerule=0mm, leftrule=.75mm, bottomrule=.15mm, rightrule=.15mm, title=\textcolor{quarto-callout-tip-color}{\faLightbulb}\hspace{0.5em}{Tip}, opacitybacktitle=0.6, left=2mm, colback=white]

Tomorrow we'll refactor into functions + modules and add better type
inference.

\end{tcolorbox}

\section{Thank You!}\label{thank-you}




\end{document}
